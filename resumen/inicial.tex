% -*-inicial.tex-*-
% Este fichero es parte de la plantilla LaTeX para
% la realización de Proyectos Final de Carrera, protejido
% bajo los términos de la licencia GFDL.
% Para más información, la licencia completa viene incluida en el
% fichero fdl-1.3.tex

\begin{center}
\Large \textbf{Resumen del Proyecto: Una aproximación a la Telefonía 2.0: Asterisk™}\\

\vspace*{1cm}

\large Manuel Camargo Lominchar, Manuel Palomo Duarte, Juan Manuel Dodero Beardo

\vspace*{0.6cm}

\normalsize \textit{Av. de la Música, 14 Piso 113 Tlfn: (+34)638000836 Correo: sir.louen@gmail.com}

\normalsize \textit{Escuela Superior de Ingeniería. C/ Chile, 1. 11002 -
  Cádiz}
\small

\vspace*{1cm}

\begin{minipage}{14cm}
\textbf{Extracto:}

Este proyecto se ha creado con la idea de poder diseñar la mayor fuente de información sobre Asterisk en Español con la intención de aplicar los medios de documentación mas eficientes posibles, y poder canalizarlo a través de un sistema eminentemente práctico. Para ello se han utilizado tres mecanismos dispares: Documentación Teórica a través del sistema MediaWiki basado en la estructura WIKI, divido por las principales temáticas que cubre la plataforma Asterisk, acompañado de una Documentación Práctica, siguiendo el Método del Caso para el desarrollo de un Caso de Estudio tratando de cubrir la mayor parte de la teoría ofrecida en la WIKI, y una Máquina Virtual trabajando bajo la solución KVM, y emulando al servidor de comunicaciones de la Empresa Ficticia creada a partir del Caso de Estudio, y con la intención de servir como modelo de demostración, desarrollo y pruebas aplicado a la realidad. 

\vspace*{0.6cm}

\textbf{Palabras clave:}

asterisk, wiki, pbx, voip, comunicaciones unificadas
\end{minipage}
\normalsize
\end{center}

% \vspace*{1cm}

\indiceArt