% -*-previo.tex-*-
% Este fichero es parte de la plantilla LaTeX para
% la realización de Proyectos Final de Carrera, protejido
% bajo los términos de la licencia GFDL.
% Para más información, la licencia completa viene incluida en el
% fichero fdl-1.3.tex

\section*{Agradecimientos}

Sin ayuda de las siguientes personas este Proyecto no hubiera podido ser realizado, por ello doy los siguientes agradecimientos...

...a mi madre, por haberme enseñado la principal habilidad para que esto se haya podido materializar, la lectura.

...a mi padre, por haberme mostrado el primer principio básico de la telefonía, el habla.

...a Lucía por haber potenciado el segundo y último principio fundamental, la escucha.

...a Sergio Serrano, Rosa Atienza y Elio Rojano, por haberme introducido en la curiosa senda de la que trata este Proyecto.

...a todos los componentes de mi equipo privado de testing, destacando a Alejandro Díaz, por la parte que les corresponde.

...y a Manuel Palomo, por haberme ofrecido el camino más interesante para plasmar mi idea en forma de Proyecto.

\cleardoublepage

\section*{Licencia}

Este documento ha sido liberado bajo Licencia GFDL 1.3 (GNU Free
Documentation License). Se incluyen los términos de la licencia en
inglés al final del mismo.

Copyright © 2012 Manuel Camargo Lominchar.

Permission is granted to copy, distribute and/or modify this document under the
terms of the GNU Free Documentation License, Version 1.3 or any later version
published by the Free Software Foundation; with no Invariant Sections, no
Front-Cover Texts, and no Back-Cover Texts. A copy of the license is included in
the section entitled ``GNU Free Documentation License''.

\cleardoublepage

\section*{Notación y formato}

Aquí incluiremos los aspectos relevantes a la notación y el formato a
lo largo del documento. Para simplificar podemos generar comandos
nuevos que nos ayuden a ello, ver \texttt{comandos.sty} para más
información. 

Cuando nos refiramos a un programa en concreto, utilizaremos la
notación: 

\programa{asterisk}.

Cuando nos refiramos a un comando, o función de un lenguaje, usaremos
la notación: 

\comando{quicksort}.