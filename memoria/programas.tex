% -*-programas.tex-*-
% Este fichero es parte de la plantilla LaTeX para
% la realización de Proyectos Final de Carrera, protejido
% bajo los términos de la licencia GFDL.
% Para más información, la licencia completa viene incluida en el
% fichero fdl-1.3.tex

\section*{MiK\TeX{}  + \TeX nic Center}

Debo reconocer que por cuestiones de la vida para desarrollar este proyecto he trabajado principalmente bajo un entorno Microsoft Windows. Aún así esto no quita para utilizar grandes herramientas Open Source como son el caso de la distribución \TeX{} para Windows llamada MiK\TeX \cite{website:miktex}. 

Por otro lado, para la edición existe un IDE de \LaTeX{} muy popular en Windows y bastante completo a mi parecer, llamado \TeX nic Center, el proyecto se mantiene en la conocida web SourceForge.org, posee un licenciamiento GPL también y se integra con MiK\TeX{} a la perfección.

Finalmente para la configuración de este documento, he utilizado una plantilla especifica para PFC que me ha proporcionado mi Director de Proyecto, y para la documentación referente a las guiás de edición, he de hacer referencia a la pagina web que me ha ido guiando en el proceso. \cite{website:latex}. Gracias a esta guía he introducido cambios generales de estructura para optimizarla un poco más y facilitarme el trabajo en general.

\section*{Doxygen}

Realmente, \programa{Doxygen} \cite{website:asterisk} no es una herramienta
que vayamos a utilizar para realizar documentos \LaTeX{}
directmaente. Sin embargo, para la documentación de código si es
bastante util.

Esta herramienta realiza una documentación automática de código
fuente. Es decir, para nuestro PFC, podemos utilizar para generar la
documentación de las APIs de nuestras librerias y demás. Puede generar
esta documentación en varios formatos, y entre ellos, \LaTeX, de forma
que podemos utilizar ese código generado en nuestra memoria de forma
automática.

\section*{GNU Make}

\programa{GNU Make} es el programa de recompilación y de control de
dependencias por excelencia. Se puede utilizar para compilar proyectos
software en diversos códigos, o como en el caso de este documento,
para compilar documentos \LaTeX{} con diversas opciones.

Para más información \cite{website:voipinfo}

\section*{Dia}

\programa{Dia} es un editor de gráficos vectoriales el cual incluye
distintas plantillas para distintos tipos de gráficos, como pueden ser
UML, ERe, diagramas de flujo, esquemas Cisco de red y un larguísimo
etcétera.

Estos diagramas podemos exportarlos a diversos formatos de imagen
(\texttt{.png}, \texttt{.eps}, ...) o a formato \texttt{.tex}, como
vimos anteriormente \cite{atdg11}