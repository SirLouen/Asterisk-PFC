% -*-wiki.tex-*-
% Este fichero es parte de la plantilla LaTeX para
% la realización de Proyectos Final de Carrera, protejido
% bajo los términos de la licencia GFDL.
% Para más información, la licencia completa viene incluida en el
% fichero fdl-1.3.tex

En este capítulo desarrollaremos un poco más la idea detrás de desarrollar una plataforma teórica basada en un sistema WIKI, concrétamente el software MediaWiki, todas sus repercusiones y ventajas que aportan en general al desarrollo de este proyecto.

También daremos un recorrido por la estructura de la WIKI y así poder entender el concepto subyacente de su estructura.

\section{Sistema MediaWiki}

Para la realización del sistema WIKI se ha usado de la plataforma MediaWiki \cite{website:mediawiki} la cual esta diseñada específicamente para este propósito.

Es conveniente considerar que la edición en la misma es semejante a la de otras WIKIs y concrétamente a la más popular, la de Wikipedia dado que en general todos estas aplicaciones pertenecen a la misma fuente.

Realmente más allá de destacar como se ha realizado el despliegue del sistema, y las buenas prácticas, en esta sección vamos a tratar la estructura de la WIKI en si.

\subsection{Origen de la WIKI}

Desde el momento que tome la determinación de desarrollar la WIKI, mi primer planteamiento fue lanzarla públicamente a un servidor en Internet de mi propiedad y enmarcarla bajo un dominio que en un futuro pudiera facilitar su adecuación. La primera idea que rondaba en mi mente era el diseño gráfico de toda una referencia basada en el sistema Asterisk con alguna referencia a España, para dar a entender fielmente la intención asociada al idioma que se pretendía instaurar.

A partir de aquí realice el registro del dominio \cite{website:wikiasterisk} muy significativo para el tema que pretendía cubrir, y simultáneamente, lo asocié al servidor web que comentaba anteriormente. Con esto resuelto ya solo quedaba instalar MediaWiki y configurarla con los parámetros estándar de uso. De aquí recibiría su nombre final el proyecto WIKI: \textbf{WIKIASterisk}

El planteamiento es dejar la WIKI sin acceso público hasta la finalización integral del proyecto y conclusión favorable, y tras esto, dar acceso únicamente a editores de confianza. El método para definir editores de confianza todavía no ha quedado demasiado definido, pero el concepto de dejarla libremente pública todavía no me resulta atractivo, dado que tengo constancia que existen excesivos Bots y las exigencias de Moderación pueden resultar excesivas, al menos para preservar íntegramente contenidos de calidad.

Otro planteamiento sería el bloqueo de las páginas principales, y dejar libre acceso para crear otras páginas desarrollando nuevos temas. Solo los moderadores, y editores de confianza tendrían acceso a este tipo de edición y también esta alternativa es candidata a ser resolutiva tras la "liberación del sistema"

\section{Estructura General}

La WIKI esta dividida en 6 secciones claramente diferenciadas, como puede verse en la Figura \ref{esquema_wiki}:

\begin{enumerate}
	\item Introducción
	\item Plan de Marcación
	\item Protocolos e Interfaces
	\item Módulos Principales
	\item Conceptos Avanzados
	\item Bibliografía y Referencias
\end{enumerate}

\figura{esquema_wiki.png}{scale=1}{Estructura general de WIKIAsterisk}{esquema_wiki}{!ht}

\subsection{Fundamentos}

El apartado de Fundamentos de Asterisk, esta enfocado a dar una perspectiva general del sistema, y establecer los primeros precedentes para su funcionamiento, partiendo de la instalación del mismo, se subdivide en tres artículos principales, según figura \ref{wiki_intro}:

\begin{enumerate}
	  \setlength{\itemsep}{10pt}

	\item \textbf{Introducción}
	En este artículo se definen conceptos básicos sobre Asterisk, se trata un poco sobre su historia, creadores, y nuevos proyectos como Asterisk Scalable Communications Framework (Asterisk SCF), que está destinado a convertirse en el futuro real de Asterisk aunque todavía no ha sufrido el despegue inicial dado que se encuentra en fase de desarrollo.
	
	\item \textbf{Instalación}
	
	Como su nombre hace referencia en este articulo se definen todos los aspectos relevantes a la instalación de Asterisk en nuestro sistema operativo de elección.
	
	\item \textbf{Arquitectura}
	
	Aquí se desarrolla en detalle toda la estructura interna de Asterisk, en concepto de Arquitectura, como se ha conceptualizado el proyecto, buscando siempre el máximo nivel de escalabilidad e independencia, basada en un sistema de módulos integrables individualmente, y una interconexión entre los mismos con metodología estándar.
	
\end{enumerate}

\figura{wiki_intro.png}{scale=1}{Esquema de la sección Introductoria de la WIKI}{wiki_intro}{!ht}

\subsection{Plan de Marcación}

La base del sistema Asterisk es el plan de Marcación. Realmente es la columna vertebral y es el método para poner en funcionamiento a voluntad del desarrollador todas las posibilidades que ofrece esta plataforma de telecomunicaciones. Todo esto puede verse en la figura \ref{wiki_dialplan}

\begin{enumerate}
	  \setlength{\itemsep}{10pt}

	\item \textbf{Introducción Dialplan}
	Podría considerarse el primer capitulo introductorio a todo el concepto y distribución del Plan de Marcación. Como se fundamenta cada concepto y los primeros pasos para construir una aplicación funcional basada en lo que Asterisk puede ofrecer. 
	
	\item \textbf{Aplicaciones Básicas}
	
	Para poder poner en funcionamiento lo aprendido a través del capitulo de introducción al Plan de Marcación es necesario conocer una serie de aplicaciones que podrían considerarse de índole imprescindible. En este apartado se van desarrollando cada una de estas aplicaciones individualmente con una detallada explicación del funcionamiento de las mismas.
	
	\item \textbf{Funciones}
	
	Quizá uno de los conceptos menos evidentes y populares del sistema Asterisk, dado que aunque realmente son verdaderas funciones propiamente dichas, son consideradas como elementos internos y estáticos del sistema, que generalmente solo hacen referencia práctica a modificaciones de comportamiento o características del sistema de módulos, aunque posiblemente para entender este concepto es necesario realizar una recapitulación general de la Arquitectura del sistema.
	
	\item \textbf{AstDB}
	
	Asterisk incorpora una base de datos interna, para almacenar información poco sensible y quizá bastante dinámica basada en un sistema no relacional llamado Berkeley DB. Quizá podría considerarse como una de las alternativas más flexibles pero a su vez menos potentes. Puede servir para pequeñas ideas, pero en el largo plazo es algo relativamente poco práctico. En este articulo se desarrolla muy por encima esta idea en términos muy generales.
	
  \item \textbf{Dialplan Avanzado}
	
	Este articulo quizá pueda ser el más desubicado de toda la Wiki, pero a su vez se encuentra en el sitio apropiado. Hay que considerar que una vez que hemos aunado todos los conceptos dispares relacionados al sistema Asterisk, todas las técnicas adicionales para construir Planes de Marcación mas sofisticados hacen de Asterisk una herramienta aún más potente y práctica. También se hace una introducción teórica a los sistemas de Respuesta de Voz Interactiva, porque realmente estos últimos son más comprensibles en la práctica que sobre el papel.
	
\end{enumerate}

\figura{wiki_dialplan.png}{scale=1}{Esquema de la sección del Plan de Marcación de la WIKI}{wiki_dialplan}{!ht}

\subsection{Protocolos e Interfaces}

En esta sección se tratan de cubrir todos los elementos relacionados al intercambio de mensajes o información, por resumirlo en lineas muy generales. Por un lado contamos con los Protocolos más comunes de Asterisk, aun habiéndose descartado algunos secundarios como el SCCP, Skinny o el MGCP de Cisco, los cuales también ofrecen servicio Asterisk, pero no son los más utilizados a nivel general. Todo esto puede observarse en la Figura \ref{wiki_protocolo}

\begin{enumerate}
	  \setlength{\itemsep}{10pt}

	\item \textbf{SIP}
	En este capitulo se explica en gran detalle el protocolo más importante asociado a Asterisk y la Telefonía IP. No es una referencia para conocer el protocolo SIP en particular, sino para conocer como funciona el protocolo en relación a Asterisk y que provecho podemos sacar del mismo
	
	\item \textbf{IAX}
	
	Por otro lado tenemos un protocolo específico, creado por la misma persona que fabricó Asterisk para cubrir una serie de deficiencias propias de SIP e intentar estandarizar un protocolo que para aquel momento se encontraba demasiado disperso. Aunque hoy en día ha seguido evolucionando SIP, e IAX ha quedado relegado como protocolo específico de comunicación entre máquinas Asterisk. Todo esto con mucho más  detalle queda explicado de este apartado
	
	\item \textbf{CLI}
	
	Asterisk aporta una interfaz muy singular basada en linea de comandos, para los administradores en la cual se pueden lanzar los mismos. Pero realmente el interés de esta interfaz suele ser de carácter informativo, dado que va ofreciendo un importante nivel de detalle de eventos configurable, del sistema interno de Asterisk en tiempo real. En esta sección hacemos un muy breve resumen al respecto dado que su función en el proyecto es meramente transitiva, aunque para el día a día de un administrador de Asterisk pueda llegar a ser vital.
	
	\item \textbf{DAHDI}
	
	Estas siglas hacen referencia a Digium/Asterisk Hardware Device Interface. Es decir, una interfaz para poder dialogar con el hardware específico de Digium y compatible, en relación a Asterisk. Esto hace realmente referencia a tarjetas para conectar con la telefonía clásica. Es un mundo donde es posible profundizar bastante, pero dado que la intención del proyecto es dar una breve visión sobre estas capacidades pero no profundizar demasiado en el detalle de las mismas, se realiza un buen repaso sobre las opciones dentro de esta rama más comunes.
	
  \item \textbf{Pasarelas ATA}
	
	Las Pasarelas o Gateways, juegan el papel de las tarjetas de comunicación como las vistas en el apartado de DAHDI, pero enlazando con nuestra máquina Asterisk a traves del protocolo SIP. Existen cientos de Pasarelas de este tipo, con muchos submodelos a su vez. Esta sección esta destinada a poder ofrecer una guía y un manual claro y conciso de la funcionalidad que aportan las pasarelas más reconocidas en el mercado actual.
	
\end{enumerate}

\figura{wiki_protocolo.png}{scale=1}{Esquema de la sección de los Protocolos e Interfaces de la WIKI}{wiki_protocolo}{!ht}

\subsection{Sistema de Módulos}

Siguiendo el esquema establecido en la Arquitectura de Asterisk, considerando que Asterisk se fundamenta principalmente en el despliegue de módulos que van cubriendo las necesidades especificas de forma autónoma, en este apartado se explican con gran detalle los módulos que conforman la base de las Comunicaciones Unificadas que ofrecen Asterisk, en un término nivel medio, considerándolos todos como fundamentales pero a su vez muy específicos en función de las distintas combinaciones de instalaciones posibles. En detalle puede verse un extracto en la Figura \ref{wiki_modulos}

\begin{enumerate}
	  \setlength{\itemsep}{10pt}

	\item \textbf{Buzones de Voz}
	Tratamos con mucho detalle todos los aspectos relevantes al sistema de Buzones de Voz, tanto a nivel de configuración como de gestión. Este es uno de los apartados que suele ofrecer más funcionalidad dentro del sistema Asterisk pero que suele quedar relegado si no aplicamos un nivel de configuración razonable para lo que ofrece, por eso aquí detallamos como alcanzarlo.
	
	\item \textbf{MeetMe}
	
	Aquí hacemos referencia al sistema de Conferencias ofrecido por Asterisk llamado MeetMe, que realmente sigue destacando por su uso. Esta sección es muy probable que sufra modificaciones en un futuro próximo dado que ya también hacemos alguna mención a las nuevas tendencias en este entorno dentro del Artículo
	
	\item \textbf{Colas de Llamadas}
	
	Este capitulo esta orientado al mundo de los Call Centers como primeros y fundamentales usuarios de la estructura de colas de llamadas como tal, y agentes atendiendo a las llamadas entrantes. Hacemos especial mención a varios aspectos derivados de este tema tales como las Estrategias en la entrada de llamadas, sistema de prioridades, y aspectos generales que hacen referencia a este mundo particular.
	
	\item \textbf{Monitorización de Llamadas}
	
	En Asterisk el termino monitorización hace referencia al hecho de estar conectado activamente a un canal de audio. En esta sección se tratan de definir los mecanismos de grabación y escucha activa de llamada que ofrece nuestro sistema.
	
  \item \textbf{Música en Espera}
	
	El concepto de Música en Espera (MOH), también esta externalizado como un módulo en Asterisk y es relativamente configurable. Solamente damos unas pinceladas sobre el tema, aunque es algo que puede resultar relativamente extenso en función de las posibilidades que podamos encontrar en esta funcionalidad
	
	\item \textbf{FAX}
	
	Los sistemas de FAX pueden haber quedado ya prácticamente obsoletos, pero aun así se siguen utilizando. Asterisk ofrece la posibilidad de configurarlos en mayor o menor medida, y aquí se explican algunas formas para realizar este cometido.
	
	\item \textbf{Codecs y Formatos}
	
	Una breve descripción sobre la gran mayoría de codecs y formatos disponibles en el sistema Asterisk
	
	\item \textbf{Asterisk Realtime}
	
	Aquí se detalla con gran precisión, como Asterisk es capaz de operar con bases de datos, y concrétamente una capacidad que tiene específica de llevar el control de la configuración interna del sistema directamente desde tablas dinámicas basadas en bases de datos de tipo relacional.
	
\end{enumerate}

\figura{wiki_modulos.png}{scale=1}{Esquema de la sección de Módulos Asterisk}{wiki_modulos}{!ht}

\subsection{Conceptos Avanzados}

Una vez que se ha cubierto la gran parte de los conceptos cruciales para el buen entendimiento del sistema Asterisk, es necesario dotar la WIKI de un apartado con conceptos más avanzados en el aprendizaje que suelen surgir la necesidad en etapas más desarrolladas. Entre estos conceptos se incluyen ideas sobre el control, sobre la capacidad de desarrollar nuevos aspectos propios, y capacidades para interconectar sistemas externos a Asterisk, con este de forma eficaz. \ref{wiki_avanzado}

\begin{enumerate}
	  \setlength{\itemsep}{10pt}

	\item \textbf{Seguridad}
	Es importante tratar todos los aspectos relevantes a la seguridad ejemplar de un sistema integral, dentro de un apartado específico. Quizá podían haber quedado dispersos en cada una de las secciones correspondientes, pero con este apartado se pretende dar una visión más orientada a la auditoría que a las buenas prácticas.
	
	\item \textbf{Generación Automática de Llamadas}
	
	Cuando pensamos en la capacidad de generar llamadas de forma automática, nos suele venir a la mente intenciones orientadas al Marketing, y a la posibilidad de ofrecer algo de forma automatizada. Pero nada más fuera de la realidad aquí explicamos muchas de las funciones que podemos utilizar con este sistema que provee Asterisk para automatizar una serie de procesos relativos entre la telefonía y la computación.
	
	\item \textbf{Registro Llamadas y Eventos}
	
	Asterisk establece un rigoroso control de varios aspectos, entre ellos incluso, del flujo de llamadas y eventos que van ocurriendo, tal y como podría ocurrir con otros tipos de servidores como los dedicados a ofrecer servicios Web, pueden almacenar todas las peticiones de entrada y salida al servidor. Con esto se pueden establecer mecanismos de control, facturación entre otros, especialmente interesantes desde una perspectiva eminentemente práctica.
	
	\item \textbf{TTS y ASR}
	
	TTS hace referencia a Text-To-Speech, es decir, Texto a Voz, o Síntesis de Voz, y ASR, significa Automátic Speech Recognition, o reconocimiento de Voz Automática. Podrían considerarse las interfaces de Voz entre la máquina Asterisk, y las personas que pudieren utilizarlas. Estos dos sistemas en esencia vienen explicados en profundidad haciendo referencia a las múltiples alternativas que nos podemos encontrar en este entorno
	
  \item \textbf{AGI}
	
	El sistema AGI, Asterisk Gateway Interface, esta orientado a poder ofrecer un soporte externo de programación en múltiples tipos de lenguajes de programación, para desarrollar aplicaciones específicas que interactúen con los flujos de llamadas dentro de Asterisk, con mecanismos relativamente sencillos de implementar y muy orientados a un entorno de desarrollo para mejorar la funcionalidad intrínseca del sistema
	
	\item \textbf{AMI}
	
	Asterisk Manager Interface, es una interfaz modular, totalmente independiente de Asterisk, que establece un tipo de Pasarela a través de la cual puede recibir comandos para invocar aplicaciones dentro del sistema Asterisk, y simultáneamente vuelca información en la misma, accesible desde el exterior y proveniente específicamente de todos los módulos que están interconectados dentro de la estructura de nuestra máquina de telecomunicaciones.
	
	\item \textbf{Intrefaces Web}
	
	Por cada uno de los módulos explicados, podría decirse que pueden existir una o varias interfaces cuya función principal es cubrir una serie de aspectos relativos a los mismos, como la configuración, el control y la gestión o la monitorización de los eventos que suceden en ellos. Dado que en el Caso de Estudio, se plantean como alternativas interesantes, el uso de determinados Interfaces Web, aprovechamos para desarrollarlos de manera práctica principalmente con algunos aspectos teóricos específicos de los mismos.
	
\end{enumerate}

\figura{wiki_avanzado.png}{scale=1}{Esquema de la sección de Aspectos Avanzados del Sistema}{wiki_avanzado}{!ht}

\subsection{Bibliografía y Enlaces de Interés}

El último apartado simplemente esta destinado a realizar referencias a determinadas páginas de interés general dentro del mundo Asterisk, otro tipo de páginas que hayan podido tener alguna relación con WIKIAsterisk, y finalmente, medios de comunicación para establecer contacto con el autor de este proyecto. \ref{wiki_bibliografia}

\begin{enumerate}
	  \setlength{\itemsep}{10pt}

	\item \textbf{Asterisk: The Definitive Guide}
	Este libro es por muchos considerado, la Biblia de Asterisk y podría decirse que gran parte de esta WIKI se fundamenta en este libro. Además es el libro recomendado por la mayoría de los expertos. Una de las pretensiones de esta Wiki es la de superar algún día los contenidos de este libro tanto en calidad como en cantidad con el resto de las aportaciones de valor.
	
	\item \textbf{Github Asterisk-PFC}
	
	Dado que el proyecto ha sido accedido desde diversas máquinas, entre las cuales se encuentran, el ordenador de trabajo y el servidor donde se aloja la máquina virtual del proyecto, y además con la idea de poder realizar un seguimiento de las distintas ediciones que ha ido sufriendo el proyecto desde las primeras instancias, se origino el planteamiento de alojar la mayor parte de los ficheros más relevantes al mismo, dentro de un sistema de control de versiones, y el elegido fue GIT \cite{website:git}, dentro ampliamente conocido repositorio \textbf{Github} \cite{website:github}.
	
	\item \textbf{Guía Wikimedia}
	
	Una de las guías que más útil me resultó para introducirme en el mundo de la edición de textos en formato Wiki, y en el que se sustenta todo el desarrollo de WikiAsterisk, pasando desde el diseño estructural, hasta las plantillas para simplificar la labor y modelación.
	
	\item \textbf{Contacto}
	
	Varios medios de comunicación para ponerse en contacto conmigo como Autor principal y único de este proyecto.
	
\end{enumerate}

\figura{wiki_bibliografia.png}{scale=1}{Esquema de la sección de Bibliografía y Enlaces de Interés}{wiki_bibliografia}{!ht}

\section{Estructura de los Artículos}

El planteamiento establecido para desarrollar los artículos estuvo basado en una Estructura general definida en la mayoría de los artículos fundamentada en la siguiente base:

\begin{enumerate}
	  \setlength{\itemsep}{10pt}

	\item \textbf{Descripción corta} 
	
	Una breve descripción introductoria para el articulo explicando brévemente la posible relación con Asterisk, o definiendo el acrónimo en caso que existiese
	
	\item \textbf{Tabla de Contenidos} 
	
	En todos los artículos existe un índice para marcar una guía muy especialmente útil en caso de querer referenciar determinadas posiciones específicas del mismo sin tener que mirar el artículo completo
  
	\item \textbf{Introducción} 
	
  Aún llamado de diferentes formas, todos los artículos poseen una primera sección introductoria que sirve para empezar a desarrollar el tema en profundidad empezando a sacar posibles subdivisiones del mismo, o enumerado funcionalidades
 
  \item \textbf{Contenido Principal} 
	
	Es la masa crítica del Articulo
  
	\item \textbf{Referencias} 
	
	Podrían considerarse, los "pie de página" de cada artículo, las cuales se encuentran dispersas por todo el Contenido Principal con un pequeño enlace que las referencia. Aquí no se ha tomado forzosamente un idioma de referencia dado que se considerarían enlaces externos

 	\item \textbf{Véase También} 
	
	Serían enlaces a otros artículos de la Wiki, teniendo siempre en mente, que absolutamente todo, tiene alguna relación con otro articulo. La idea de esto, es que fuera viable poder establecer como mínimo un camino de lectura, y poder recorrer el 100\% de los artículos sin tener que visitar la página principal.
	
	\item \textbf{Enlaces Externos} 
	
	Todas las empresas, documentos interesantes y direcciones que pudiera hacer referencia directamente al artículo y cobraran sentido con respecto al mismo. Podría considerarse un tipo de bibliografía añadida al artículo conjunta al grupo de Referencias.

\end{enumerate}

 \subsection{Consideraciones de Formato y Estructura}

 En el momento de crear un artículo se diseñaba principalmente aplicando un Artículo de ejemplo:

 http://wikiasterisk.com/index.php?title=Estructura

 El cual contenía la estructura anteriormente comentada en líneas generales, y utilizaba algunas plantillas prácticas para el desarrollo del mismo.

 También se creo una categoría específica para englobar todas las Plantillas posibles en caso que pudiera requerirse de su uso: 

 http://wikiasterisk.com/index.php?title=Categor\%C3\%ADa:Principal:Plantillas

 Dentro de la Estructura existen dos indicadores específicos del estado del Artículo. Para mi criterio les he dado el siguiente uso:

 \begin{itemize}
	  \setlength{\itemsep}{10pt}

 \item \textbf{To Do}
 
 En caso que el Artículo no estuviese terminado, o tuviera intención de ampliarlo en un futuro, se marcaría con esta Plantilla, aque ademas asigna una Categoría especial, para poder sacar en un futuro un listado de todos los artículos marcados y así poder seguir trabajando donde lo dejamos. Se representa con una pequeña caja como podemos ver en la figura \ref{todo}

\figura{todo.png}{scale=1}{Indicador de Artículo pendiente de Completar o Mejorar}{todo}{!ht}

\item \textbf{Formato}

Por otro lado, en caso que un Artículo no haya cumplido toda la estructura anteriormente mencionada al completo, incluyendo al menos un enlace de cada tipo, y como nota adicional, al menos una imagen relativa al contexto (para resultar más gráficamente descriptivos de manera general), quedaba marcado con una caja representativa indicando que le falta cumplir las reglas de Formato preestablecidas. La representación de la misma puede verse en la Figura \ref{format}

\figura{format.png}{scale=1}{Indicador de Artículo pendiente Formatear}{format}{!ht}

\end{itemize}