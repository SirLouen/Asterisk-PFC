% -*-caso_de_uso_2.tex-*-
% Este fichero es parte de la plantilla LaTeX para
% la realización de Proyectos Final de Carrera, protejido
% bajo los términos de la licencia GFDL.
% Para más información, la licencia completa viene incluida en el
% fichero fdl-1.3.tex

\newpage

\color[rgb]{0,0,1}

\section{Mejorando lo presente}

Tras haber realizado los trabajos solicitados, en la siguiente reunión con Zutano Doe, tuvimos la oportunidad de conocer su experiencia. Nosotros sabíamos que cuando viera los resultados querría más, ya que quedaba latente el hecho, que nuestra máquina Asterisk aun tenía mucho potencial por desarrollar.

Uno de los aspectos que notó el Sr. Doe, es que evidentemente, aún no podían cursar llamadas. Era lógico que nosotros no íbamos a ser tan atrevidos como para desarrollar un plan de marcación especifico sin antes consultar su opinión dado que aquí estaríamos entrando en el tema económico de facturación, con la compañía elegida UCA Telecom.

Averiguamos que Zutano, mantenía conversaciones con Alemania regularmente, ya que era el su principal proveedor de Vehículos, pero estos temas los trataba el personalmente, y no quería que nadie más pudiera cursar llamadas fuera de nuestro país. Además como es común no se fiaba que sus usuarios pudieran llamar libremente a números de tarificación extraordinaria (80X), o de pago por uso, pero sabía que los números de tarificación adicional, estaban ya demasiado extendidos en las grandes corporaciones, así que no podía de momento evitar su uso (90X). También era consciente, que la mayor parte de los clientes hoy en día, utilizan el móvil como su principal medio de comunicación, asi que quería las llamadas a móviles totalmente libres. Un último aspecto que hizo hincapié, es en la posibilidad de que los Mánager pudieran llamar a números de información telefónica (118XX), en caso que cualquiera de sus trabajadores requiriese la información. Le expresamos, que no nos pareció demasiada buena idea, ya que son bastante caros, y hoy en día la mayor parte de esa información esta e Internet, pero aún así, insistió disponer de esto, exclusivo a los Jefe de departamento. Finalmente le hicimos el apunte, que dado la falta de regulación en la actualidad de la telefonía IP, los números de emergencia, podían fallar, por eso aunque habilitaríamos el 112 no se aseguraba su correcto funcionamiento. Tampoco le afecto demasiado, ya que todos sus empleados disponían como mínimo, de su teléfono móvil particular y en el caso de los jefes un móvil de empresa.

Ya que su negocio empezaba a marchar, quería empezar a publicitar un número común genérico, para ofrecérselo a los clientes y poder utilizarlo con fines de Marketing, por ejemplo en sus recientes anuncios en el periódico local. Este teléfono principal, tendría que tener un sistema de interacción con el cliente más personalizado, es decir, la implementación de un IVR (Interactive Voice Response). El Sr. Doe todavía no tenía muy claro como debía operar ya que todavía no estaban sentados los cimientos de la empresa en profundidad. Nosotros le propusimos de momento la siguiente idea:

\begin{enumerate}

\item Al entrar la llamada, lanzar un mensaje de Bienvenida
\item A continuación, ofrecer numéricamente las posibilidades de contactar con cada departamento
\item Si no pulsa nada, volver a ofrecerle todas las posibilidades
\item Si sigue sin pulsar nada, entonces dirigirle al departamento de Marketing

\end{enumerate}

Nosotros tenemos un horario de atención al cliente restringido, por tanto sería conveniente, que si nos llaman, fuera de nuestro horario, en vez de aparecer este IVR, apareciera un mensaje que indique que no podemos atenderle, ofreciendo nuestro horario y la posibilidad de dejar un mensaje en un buzón de voz, por ejemplo del Responsable de Marketing. El horario de apertura que nos ofreció el gerente sería de Lunes a Viernes de 9h a 20h ininterrumpidas

Por último, ya que sacamos el tema de los buzones, también estaba interesado en poder disponer de Buzones de Voz para todos su Jefes de Departamento y para él. Todas los mensajes dejados en los contestadores irían a parar a sus correos en cualquier formato de Audio, pero además tendría que existir la posibilidad de ser almacenados en el sistema y escuchados a voluntad por cada uno de los usuarios dentro del sistema. El Sr. Zutano, adelantándose, tenía constancia que Asterisk era de origen americano, así que sugirió, que que si el menú del sistema de buzón de voz estuviese en Español sería un punto muy favorable, ya que no todos sus empleados estaban duchos en el manejo del Inglés.

En principio todo este planteamiento le resulto aceptable y necesario de poner en marcha cuanto antes, aunque seguramente sujeta a modificaciones en un futuro. Siendo realistas, hasta aquí es lo que suelen ofrecer el 99\% de las centralitas de telefonía ofrecidas para cualquier PYME.

\newpage

\color[rgb]{0,0,0}

\subsection{Preparando las Llamadas Salientes}

Vamos a mejorar el anterior Plan de Marcación existente. A continuación muestro el nuevo código modificado sobre el anterior.

Diseccionándolo en primer lugar podemos ver el nuevo contexto [exterior]. Con este contexto especificamos siguiendo los patrones específicos del Plan de Marcación definimos las posibilidades de marcación que podemos hacer para realizar llamadas a números Nacionales (tanto aquellos que empiezan por 8 como por 9), Móviles y números de información (restringiendo exclusivamente a 900, 901 y 902). 

\begin{lstlisting}[style=bash,title={/etc/asterisk/extensions.conf}]

[exterior]
; --- Contexto Especifico para Llamadas al exterior

exten => _[89]ZXXXXXXX,1,NoOp()
same => n,Dial(SIP/proveedorip/${EXTEN})
same => n,Hangup()

exten => _6XXXXXXXX,1,NoOp()
same => n,Dial(SIP/proveedorip/${EXTEN})
same => n,Hangup()

exten => _90[0-2]XXXXXX,1,NoOp()
same => n,Dial(SIP/proveedorip/\${EXTEN})
same => n,Hangup()

\end{lstlisting}

Por otro lado, dentro de los contextos de Mánager y gerencia, podemos ver también los requisitos específicos a los que solo podrán acceder los mismo como vimos en el primer DialPlan, por el sistema jerárquico creado gracias a la utilización de los ``include''. Con estos específicamente, definimos marcaciones para números de información (Aquellos que empiezan por 118), y para el gerente llamadas a números internacionales de destino Alemania, que tienen como peculiaridad, que aunque tienen una marcación común de inicio, a partir del 5 dígito, la numeración es variable considerando a continuación el prefijo regional + el número local (de 3 a 9 dígitos).

\begin{lstlisting}[style=bash,title={/etc/asterisk/extensions.conf}]

; --- Contexto especifico de los Managers
[manager]

include => extensiones

; --- Llamadas a numeros de informacion
exten => _118XX,1,NoOp()
same => n,Dial(SIP/proveedorip/${EXTEN})
same => n,Hangup()

; --- Contexto especifico del Gerente
[gerencia]

include => manager
include => grabaciones

; --- Los numeros en Alemania son de longitud variable
; --- Para evitar plantear todas las posibilidades
; --- Hacemos uso del comodin "."

exten => _0049.,1,NoOp()
same => n,Dial(SIP/proveedorip/${EXTEN})
same => n,Hangup()

\end{lstlisting}

En siguiente lugar, dentro del contexto [entrantes], hemos añadido un nuevo número Direct Dial-In que redirige a la extensión \textbf{start} considerando que sea la genérica para este contexto. En esta extensión vamos a crear nuestra Operadora Automática, tal y como esta definida en el caso de uso. Para ello vamos utilizando una combinación de Aplicaciones según su conveniencia. La idea subyacente es la siguiente y puede verse en la figura \ref{operadora1}:

\begin{enumerate}

\item Ponemos un contador a 0, la idea es que cuando hayamos pasado dos veces por la operadora sin resultado, desviemos la llamada según especificado en el paso número 4
\item Lanzamos el mensaje de bienvenida.
\item Comprobamos si el horario actual coincide con el horario de apertura o no. 
\item En caso que no coincida, lanzamos los mensajes de ``Cerrado'' y a continuación el Buzón de Voz del responsable de Marketing
\item En caso que si coincida, nos vamos a la etiqueta ``bienvenida'' y a partir de ahí lanzamos el menú contextual para dar la opción a marcar una u otra opción. 
\item Si no pulsa nada pasado el mensaje + 1 segundo, desviamos a la extensión ``relative timeout'' 
\item Desde aquí comprobamos cuantas veces hemos pasado por aquí, si solo fue una vez, entonces volvemos al principio del algoritmo
\item En cambio si fueron ya dos veces, mandamos la extensión a Marketing.

\end{enumerate}

\figura{operadora1.png}{scale=1}{Flujo de Llamada de la Operadora Automática}{operadora1}{!ht}

\begin{lstlisting}[style=bash,title={/etc/asterisk/extensions.conf}]

; Operadora Automatica
exten => s,1,NoOp()
same => n,Set(ivrcont=0)
same => n,Answer()
same => n(bienvenida),Playback(bienvenida)
same => n,GotoIfTime(09:00-20:00,mon-fri,*,*?abierto)
same => n,Playback(cerrado&horario)
same => n,VoiceMail(1@market)
same => n,Hangup()
same => n(abierto),Background(menu-ivr)
same => n,WaitExten(1)

exten => t,1,NoOp()
same => n,Set(ivrcont=$[${ivrcont}+1])
same => n,GoToIf($[${ivrcont}<2]?s,bienvenida:0,1)

; 1 a Ventas
exten => 1,1,Goto(extensiones,21,1)
; 2 a Almacen 
exten => 2,1,Goto(extensiones,31,1)
; 3 a Posventa
exten => 3,1,Goto(extensiones,41,1)
; 0 a Marketing
exten => 0,1,Goto(extensiones,51,1)

\end{lstlisting}

Finalmente, como podía verse anteriormente, hemos modificado todas las extensiones especificas de los Mánager, para adaptarlas al uso de Buzón de Voz. Nótese, que hemos añadido una restricción temporal en la aplicación de llamada \textbf{Dial} a 30 segundos, pasados los cuales, se ejecutaría la aplicación de grabación de mensajes de Buzones de Voz. . Por último hemos añadido extensiones especificas para el acceso a los buzones de sus respectivos usuarios, marcando primero el 9 y a continuación el número de la extensión.

\begin{lstlisting}[style=bash,title={/etc/asterisk/extensions.conf}]

; --- Plan General de Extensiones
[extensiones]

include => exterior

exten => 10,1,NoOp()
same => n,Dial(SIP/10,30)
same => n,VoiceMail(0@admin)
; Acceso al menu de Buzon de Voz
exten => 910,1,VoicemailMain(0@admin)

exten => 11,1,NoOp()
same => n,Answer()
same => n,Set(CHANNEL(tonezone)=starwars)
same => n,Dial(SIP/11,30)
same => n,VoiceMail(1@admin)
same => n,Hangup()
; Acceso al menu de Buzon de Voz
exten => 911,1,VoicemailMain(1@admin)

exten => 12,1,NoOp()
same => n,Dial(SIP/12)

\end{lstlisting}

Dado que la mayor parte de las pistas de audio del sistema Asterisk se encuentran en inglés, y por tanto dificultaría bastante la gestión de los Buzones de Voz por parte de los usuarios, así como determinados mensajes (como los mensajes por defecto para dejar una grabación) para los posibles llamantes en el territorio nacional, hemos tomado la iniciativa de instalar un paquete de voces profesional, publicado por http://www.voipnovatos.es, en formato ALAW aún pesados, muy ágil para nuestro sistema. \footnote{Sonidos Principales, http://www.voipnovatos.es/voces/voipnovatos-core-sounds-es-alaw-1.4.tar.gz} \footnote{Sonidos Adicionales, http://www.voipnovatos.es/voces/voipnovatos-extra-sounds-es-alaw-1.4.tar.gz}

La configuración especifica del fichero voicemail.conf para que el sistema de Buzones de voz funcione dentro de nuestras pretensiones se detalla a continuación. Para más información podemos consultar el apartado especifico dentro del Wiki. \footnote{Buzones de Voz, Wikiasterisk, http://wikiasterisk.com/index.php?title=Buzones\_de\_Voz}:

\begin{lstlisting}[style=bash,title={/etc/asterisk/voicemail.conf}]

[general]
format = wav49|gsm|wav
serveremail = mcamargo@wikiasterisk.com
attach = yes
maxmsg = 50
maxsecs = 120
minsecs = 5
skipms = 3000
maxsilence = 10
silencethreshold = 128
maxlogins = 2
pbxskip = no
fromstring = VoiceMail Asterisk-PFC
emaildateformat = %A, %B %d, %Y at %r
minpassword = 4
backupdeleted = 20

\end{lstlisting}

Un dato curioso, la cabecera del fichero voicemail.conf, hace referencia a una generación automática. Esto es debido a que considerando que hemos puesto la contraseña por defecto en muchos buzones, idéntica al número de buzón, por defecto al ejecutar la aplicación VoiceMailMain, solicita el cambio de contraseña. Al realizar este cambio, la aplicación AppVoicemail, hace un cambio en este fichero de forma automática para adaptar las nuevas contraseñas prefijadas. En este caso sería conveniente volver a restablecer las contraseñas a las suyas por defecto como el resto, pero lo dejare en este estado, para que sirva a efectos informativos:

\begin{lstlisting}[style=bash,title={/etc/asterisk/voicemail.conf}]

;!
;! Automatically generated configuration file
;! Filename: voicemail.conf (/etc/asterisk/voicemail.conf)
;! Generator: AppVoicemail
;! Creation Date: Tue May 22 15:02:20 2012
;!

\end{lstlisting}

Los datos relativos al mensaje de la zona, suelen ser datos genéricos utilizados en la mayor parte de las instalaciones, basados en las preferencias reconocidas depediendo la costumbre de la sociedad en cada región del mundo:

\begin{lstlisting}[style=bash,title={/etc/asterisk/voicemail.conf}]

; Parametros por defecto del sistema de mensajes de Asterisk
[zonemessages]
eastern = America/New_York|'vm-received' Q 'digits/at' IMp
central = America/Chicago|'vm-received' Q 'digits/at' IMp
central24 = America/Chicago|'vm-received' q 'digits/at' H N 'hours'
military = Zulu|'vm-received' q 'digits/at' H N 'hours' 'phonetic/z_p'
european = Europe/Madrid|'vm-received' a d b 'digits/at' HM

\end{lstlisting}

Finalmente para la definición de cada uno de los buzones, hemos elegido algunos de ejemplo para mostrar el nivel de configuración al que pueden llegar a someterse:

\begin{lstlisting}[style=bash,title={/etc/asterisk/voicemail.conf}]

[admin]
0 => 0,John Doe,jdoe@wikiasterisk.com,,tz=european|
dialout=extensiones|sendvoicemail=yes|callback=extensiones|
review=yes|forcename=yes|forcegreetings=yes

1 => 1,Robert Mackencie,rmackencie@wikiasterisk.com,,tz=european|
dialout=extensiones|sendvoicemail=yes|callback=extensiones|
review=yes|forcename=yes|forcegreetings=yes

[ventas]
1 => 1,Hugh Douglas,hdouglas@wikiasterisk.com,,tz=european|
dialout=extensiones|sendvoicemail=yes|callback=extensiones|
review=yes|forcename=yes|forcegreetings=yes

\end{lstlisting}

Para el envío de e-mail es necesario tener instalado algún servidor de envío de correo. En este caso, durante la instalación gracias a la herramienta tasksel habíamos instalado Postfix, y LAMP que nos serían de gran ayuda en el transcurso de la programación de esta central. En este caso Postfix será el encargado de enviar los mensajes de e-mail a nuestros destinatarios de correo según configuración en el fichero voicemail.conf. Considerar que para el envio de correo es importante que el servidor de correo este bien configurado dado que la mayor parte de los servidores de Internet rechazan correos por doquier, especialmente los mas populares (GMAIL, HOTMAIL, etc). Dentro del fichero /var/log/mail.log podemos ver el resultado del envío en caso que nuestro destinatario no haya recibido el mensaje, y considerar factores por los cuales nuestro servidor de correo Postfix deba ser mejorado. En este sentido es conveniente que nuestra conexión a Internet este asociada a una dirección IP estática, ya que la mayoría de las direcciones dinámicas han podido ser introducidas en las listas negras como http://www.spamhaus.org.

En este caso, dado que no disponemos de IP estática, hemos utilizado un servidor SMTP remoto (smarthost) para configurar nuestro servidor Postfix. Toda la configuración relativa que no atañe directamente a este proyecto podrá verse dentro de la máquina virtual en el fichero principal de configuración del servidor /etc/postfix/main.cf. \cite{website:relaysmtp}

\subsection{Resolución:Preparando las Llamadas Salientes}

Aunque tuvimos que hacer una modificación en el Plan de Marcación en el apartado anterior para comprobar que las extensiones especificas para llamadas a través del proveedor IP UCATelecom funcionaban, ahora si es posible hacer la prueba directamente y comprobar a través de la interfaz CLI que se efectúa la llamada saliente, y esta regresa entrando por uno de nuestros DDI. En este caso, con cualquier extensión de las creadas en apartados anteriores, probamos a llamar de nuevo al numero 956001121. En este caso, si tenemos configurado también la extensión 21, debería empezar a sonar de inmediato.

Podemos comprobar que esto se esta efectuando correctamente si obtenemos el resultado de la interfaz CLI como puede verse en la figura \ref{ejemplo_cli_sip_2}, y en caso que no aparezca correctamente esta salida, podremos observar en que punto esta fallando concrétamente.

\figura{ejemplo_cli_sip_2.png}{scale=1}{Ejemplo de salida y entrada a través de un Proveedor IP}{ejemplo_cli_sip_2}{!ht}

En el caso de las otras opciones planteadas de llamadas salientes al exterior, simplemente comprobar según el guión que los patrones de las extensiones para el resto de los casos posibles (llamadas a móviles, llamadas a números de información y números extranjeros) coinciden correctamente. Si el proveedor IP ofreciese salida al exterior real, podrían comprobarse en tiempo real, cosa que realmente no es del todo necesario, ya que podemos ver en nuestra interfaz CLI la salida que reporta el hecho de marcar a las posibles combinaciones y poder comprobar también que se estan efectuando de manera adecuada al menos en nuestro lado (independientemente de la configuración del Proveedor IP). Un ejemplo de esto, considerando que el proveedor IP no tiene configuradas las llamadas a números de información (por ejemplo 11811), en la Figura \ref{ejemplo_cli_sip_3} podemos observar como en nuestro servidor la salida si se cursa correctamente a través del proveedor, dado que entra en la extensión 118XX y va pasando prioridades, pero el proveedor acaba rechanzando la llamada porque no localiza ese patrón en su Plan de Marcación (o podrían ser otros factores, pero eventualmente ya sabemos que al menos la asociación de Patrón desde nuestro número saliente 11811 con el patron \_118XX.

\figura{ejemplo_cli_sip_3.png}{scale=1}{Ejemplo de salida sin retorno a través de un Proveedor IP}{ejemplo_cli_sip_3}{!ht}

Para comprobar que la operadora automática realiza su función el medio más eficaz sería llamando desde cualquier extensión al número Directo, 956001191, y comprobar manualmente que las distintas operaciones segun el esquema seguido en la Figura \ref{operadora1} surte efecto.

Finalmente, acerca de los buzones de voz, en el momento que llamamos a cualquier extensión que tenga asociado un buzon de voz, podemos esperar a grabar un mensaje y ver si en la consola se almacena el resultado del mismo, segun vemos en la figura \ref{ejemplo_cli_voicemail}

Adicionalmente tenemos la posibilidad de ver si existen ficheros de audio almacenados en los buzones concretos, accediendo por ejemplo, al buzón de la extensión 10 desde la ruta \textbf{/var/spool/asterisk/voicemail/default/10/INBOX/}, o de forma genérica: \\ \textbf{/var/spool/asterisk/voicemail/<contexto\_buzon>/<numero\_buzon>/INBOX/} podemos ver según la imagen que aparece un fichero de audio con el nombre msg00001.wav (o el nombre que haya reportado la Interfaz CLI de como quedó registrado).

\figura{ejemplo_cli_voicemail.png}{scale=1}{Ejemplo de salida de un mensaje en el Buzón de Voz}{ejemplo_cli_voicemail}{!ht}

\newpage

\color[rgb]{0,0,1}

\section{Interconectando máquinas Asterisk}

Una vez terminados los trabajos, volvemos una semana después a UCA Autos para comprobar que todo va bien. Casualmente nos encontramos con el Sr. Zutano el cual nos comenta que esta muy satisfecho con los trabajos realizados, y la evolución que ha supuesto para su empresa haber montado un sistema como este. Nos hace saber que tiene algunos proyectos en mente, pero todavía no se ha dispuesto a llevarlos adelante porque necesita plantear con los Jefes de Departamento sus opciones a ver la opinión de los mismos, y la mejor forma de implementarlos. 

Durante la pequeña conversación, nos dejo caer, que en breve seguramente, ampliaría el personal por departamentos, dado que las cosas le estaban yendo muy bien, y cada día sus necesidades aumentaban. Esto sería relevante para nosotros por el hecho de intentar pasar de una central específica a lo más genérica posible para evitar complicarnos en un futuro.

Justamente pocos minutos después, recibimos una llamada de nuestro contacto técnico en UCA Telecom, informándonos que desde hace apenas unas horas han implementado en sus servidores la nueva funcionalidad de interconexión a través de un protocolo especifico para sistemas Asterisk llamado IAX. Una alegría para nosotros dado que todas nuestras instalaciones son Asterisk, y eramos conscientes que no estábamos cumpliendo en la mayoría de las mismas a nivel de seguridad, por una cuestión de tiempo y en consecuencia coste de implementación, excepto en algunos clientes que nos lo pidieron expresamente. De hecho el gerente de UCA Autos ya nos hizo un comentario para su mejoría en un futuro, sobre algo que había escuchado de encriptación de las llamadas, para evitar que le pudieran ``pinchar'' las lineas telefónicas. Ahora con el sistema IAX teníamos la oportunidad de actualizar nuestras instalaciones en general, incluyendo el concesionario.

\newpage

\color[rgb]{0,0,0}

\subsection{Acceso al protocolo IAX}

Cara a la generalización del sistema, con vistas a una estandarización de las extensiones, hemos decidido dar un vuelco en la estructura de contextos del sistema de Buzones de Voz. Pasaremos todas los buzones tratados al contexto por defecto [default], y además pasaremos a dotar a todas las extensiones del sistema, de su respectivo buzón de voz pese a no ser una petición original, entendimos que se pensó por motivos de recorte presupuestario. Además si queremos implementar en un futuro un sistema de Directorio de extensiones, este paso será bastante conveniente. El procedimiento para el cambio fue el siguiente:

\begin{itemize}

\item En el fichero /etc/asterisk/sip.conf actualizamos todos parámetros de los buzones, poniendole nombres equivalentes al nombre del dispositivo (Y número de extension), seguidos del contexto default (@default)
\item En el fichero /etc/asterisk/voicemail.conf pasamos todos los buzones al contexto [default] y cambiamos también los nombres siguiendo el sistema anterior

\end{itemize}

Pero lo realmente interesante, fue el método de estandarización para las extensiones seguido en el fichero referente al Plan de Marcación:

\begin{lstlisting}[style=bash,title={/etc/asterisk/extensions.conf}]

; --- Macro para Extensiones
; --- Con Buzones de Voz Mejorados

[macro-telefonos]
exten => s,1,NoOp()
same => n,Set(buzon=${MACRO_EXTEN})
same => n,Set(CHANNEL(tonezone)=starwars)
same => n,Dial(SIP/${buzon},30)
same => n,Gotoif($["${DIALSTATUS}" = "BUSY"]?ocupado)
same => n,VoiceMail(${buzon},u)
same => n,Hangup()
same => n(ocupado),VoiceMail(${buzon},b)
same => n,Hangup()

[macro-buzon]
exten => s,1,NoOp()
same => n,Set(buzon=${MACRO_EXTEN:1})
same => n,VoiceMailMain(${buzon})
same => n,Hangup()

; --- Plan General de Extensiones
[extensiones]

include => exterior

; Minimizando las combinaciones para
; ofrecer justo las extensiones disponibles

; Extensiones de Administracion
exten => _1[0-4],1,Macro(telefonos)
exten => _91[0-4],1,Macro(buzon)

; Extensiones de Comercial
exten => _2[1-5],1,Macro(telefonos)
exten => _92[1-5],1,Macro(buzon)

; Extensiones de Almacen
exten => _3[1-4],1,Macro(telefonos)
exten => _93[1-4],1,Macro(buzon)

; Extensiones de Posventa y Marketing
exten => _[45][1-3],1,Macro(telefonos)
exten => _9[45][1-3],1,Macro(buzon)

\end{lstlisting}

Aquí introducimos el uso de Macros. Para ello creamos dos Macros básicas observando un poco los patrones que se repetían en el anterior DialPlan y dado que ahorrábamos en trabajo de redundancia en el tecleo de código, aprovechamos para introducir algunas mejoras al sistema.

Por un lado una macro para la llamada a las extensiones, y en caso de no respuesta a los 30 segundos (o comunicando) pasaría a la siguiente aplicación condicional, que mandaría a la aplicación para la grabación de mensajes del buzón de voz, con un parámetro especifico u otro (para lanzar el mensaje de ocupado o no disponible en función del resultado de la llamada).

Por otro lado, una macro para acceder al sistema de gestión de cada buzón de voz, pulsando el 9 seguido de la extensión que queramos consultar.

He utilizado patrones para restringir con exactitud, las extensiones solicitadas por el gerente de UCA Autos, y además cuanto más especifico sea nuestro plan, mejor nivel de seguridad alcanzamos.

Ahora atendiendo a la llamada de UCA Telecom, vamos a adaptar nuestro sistema para la comunicación a través del protocolo IAX. Nuestro contacto técnico nos planteo la posibilidad de establecer la comunicación cifrada y luego utilizando dos posibilidades más convenientes de autentificación, aplicando un reto MD5 sobre la clave elegida o utilizando claves RSA privadas y publicas (Asterisk solo permite la encriptación DES3, Estándar de Encriptación de Datos Triple). Siguiendo el mecanismo explicado en el apartado de Seguridad de la Wiki \footnote{Encriptando Comunicación IAX, WikiAsterisk, \\http://wikiasterisk.com/index.php?title=Seguridad\#Encriptando\_la\_Comunicaci.C3.B3n\_con\_RSA}, generamos nuestra clave pública y privada, con el script según hemos visto explicado, y cargamos las claves en nuestro Asterisk.

Adicionálmente, gracias a una interfaz web que nos facilita el proveedor UCA Telecom, somos capaces de introducir en su sistema, nuestra clave pública, y además nos descargamos la clave pública que nos facilitan para introducirla en nuestro directorio de Asterisk por defecto /var/lib/asterisk/keys y así poder cargar las claves en nuestro sistema.

Y con todo ya generado y preparado, creamos nuestro fichero iax.conf en el directorio por defecto de ficheros de configuración de Asterisk /etc/asterisk según podemos ver a continuación:

\begin{lstlisting}[style=bash,title={/etc/asterisk/iax.conf}]

[general]
bindport=4569
srvlookup=yes
language=es
encryption=yes
; Si usamos claves RSA es conveniente desactivar 
; esta opcion.
forceencryption=no
trunk=yes
requirecalltoken=yes

\end{lstlisting}

En primer lugar, definimos algunos parámetros generales, que destacadamente, hacen referencia al hecho específico de realizar algun tipo en encriptación de la comunicación, y forzar opciones de seguridad más restrictivas.

A continuación planteamos los dos comportamientos por separado, para definir las claves privadas y publicas RSA, de forma independiente, para conectar con el proveedor de telefonía UCA Telecom:

\begin{lstlisting}[style=bash,title={/etc/asterisk/iax.conf}]

[ucatelecom]
type = user
host = dynamic
auth = rsa
inkeys = ucatelecom
context = entrantes

[proveedorip]
type = peer
outkey = ucaautos
username = ucaautos
host = home-asterisk.local
auth = rsa
qualify = yes

\end{lstlisting}

Por otro lado y para terminar, en el fichero de configuración del Plan de Marcación, solo tenemos que editar todos los puntos donde nos encontramos una comunicación a través del protocolo SIP, y pasarlo por el protocolo IAX2. 

\subsection{Resolución: Acceso al protocolo IAX}

En este caso, la comprobación es muy sencilla, dado que realmente solo hemos reformulado lo que ya teníamos hecho anteriormente, pero con un estilo más eficaz y elegante. 

Por ejemplo, podemos comprobar el efecto de una llamada a una extensión realizando la misma entre dos teléfonos, y también comprobando si el buzón de voz hace su papel como vimos en el apartado anterior.

En cuanto a la conexión IAX, ocurre exactamente lo mismo que con el proveedor IP a través de SIP. Simplemente llamando a cualquiera de los teléfonos DDI 9560011XX acabado en el número de extensión de las extensiones de Mánager deberíamos de poder recibir la llamada si la tenemos configurada en un teléfono sin problemas

Además es posible comprobar que las claves RSA están correctamente configuradas en el servidor ejecutando el comando en la interfaz CLI:

\begin{lstlisting}[style=consola]
CLI> keys show
Key Name           Type     Status           Sum
------------------ -------- ---------------- --------------------------------
ucaautos           PUBLIC   [Loaded]         a18c6e5866db067d9c378fb3cafd967c
ucaautos           PRIVATE  [Loaded]         a46b82f019beb91702a73eec276bf6b6
ucatelecom         PUBLIC   [Loaded]         c3abd8244e94ddb27d02b4b9ed9f9236
\end{lstlisting}

\newpage

\color[rgb]{0,0,1}

\section{Distribuidor de Piezas de Repuesto}

Pasado un mes desde nuestra última instalación, pudimos observar como la mayor parte del personal se sentía satisfecha con la solución de telefonía ofrecida. Algunos Mánager de departamento se habían puesto en contacto con el gerente, para consultarle la posibilidad de sacar más partido al sistema. Justamente, en este periodo, se presentó un nuevo reto a la compañía UCA Autos, dado que fue nombrada Distribuidora oficial de piezas de Repuesto para toda la provincia que representaba.

Esto significaba tener que dar servicio a toda la red de talleres, y distribuidores de recambios locales, para la Marca que representaban, pero a nivel provincial, lo que supondría tener que ofrecer un nivel de servicio a sus nuevos clientes. La comunicación se establecía tanto presencialmente, por los gestores de estos negocios, o la vía más común, pedidos a nivel telefónico, dado que aunque existía una interfaz en Internet por la cual podrían cursar sus pedidos, la mayor parte de los clientes, no tenían demasiado control sobre las nuevas tecnologías (gente de relativa avanzada edad, propietarios de talleres acostumbrados a un servicio de toda la vida), y por otro lado, también era frecuente la consulta como atención al cliente, dada la amplia selección de piezas, que causaba relativa confusión entre los compradores.

Es en este momento, cuando el Sr. Doe contactó con nosotros, para ver si existía la posibilidad de montar una pequeña plataforma de Call Center, que fuera fiable, pero a ser posible, no demasiado cara, porque según había entendido por sus compañeros colaboradores, las que ellos conocían resultaban excesivamente costosas, pero ofrecían estadísticas de uso para su análisis de calidad de servicio, y en muchos casos tuvieron que optar por soluciones muy pobres que les aportaban escasa o nula información en la mayor parte de los casos.

Tras la conversación con Zutano, le explicamos que con el sistema Asterisk que había elegido implementar, podría tener el mismo sistema que le comentaron sus compañeros, pero a coste menor aproximado que el que ellos le reportaron, dado que solo se reflejaba el coste de implantación en horas por parte de nuestra empresa. El gerente de UCA Autos, no podía dar crédito a nuestras palabras y nos pidió que lo hiciéramos lo antes posible.

Las especificaciones fueron muy sencillas. Para el departamento de almacén existían tres puestos aparte del Mánager. Quería que esos tres puestos fueran los tres componentes de la primera instancia de Call-Center, ya que de momento no tenía intención de contratar más personal y ver si podía valerse para atender a sus clientes con lo que disponía en la actualidad, o plantear alguna ampliación en un futuro. 

Por otro lado, quería tener un teléfono directo especifico para el Call-Center, al entrar la llamada, diese un mensaje de bienvenida, y pusiera al cliente en la cola de espera hasta que pudiera ser atendido por un agente. Y si fuera posible también nos pidió que el sistema dijera eventualmente al cliente en espera, su posición en la cola. Le confirmamos que sus peticiones eran todas posibles, y que nos pondríamos manos a la obra de inmediato.

Le preguntamos como quería distribuir sus llamadas entre los agentes disponibles. Hicimos algunos comentarios sobre las estrategias disponibles, y eligió la estrategia más simple de momento, dado que todavía no conocía en profundidad como resultaría esta nueva operación, que sonase por igual el teléfono en todas las lineas disponibles a cada momento.

Finalmente, también le inquietaba, el hecho de poder controlar la calidad de las llamadas. En un principio le comentamos que existía la posibilidad tanto de escuchar las llamadas en tiempo real, como poder realizar grabaciones de llamadas en disco. Ambas opciones le parecían interesantes por igual, especialmente para el Responsable de Almacen y Marketing los cuales debían hacer un seguimiento intensivo por la forma en la que la que se realizaba la comunicación pudiera optimizarse en un futuro. En consecuencia, se planteo el hecho de tener los dos servicios disponibles

\newpage

\color[rgb]{0,0,0}

\subsection{Colas en Asterisk}

En primer lugar, crearíamos dentro de la máquina Asterisk, el fichero de configuración especifico para el sistema de Colas. Pese a que existen muchos parámetros disponibles nos centraríamos en las peticiones del Sr. Zutano, y si en un futuro apreciaba deficiencias, intentaríamos ajustárselas con las mismas.

El fichero se muestra a continuación:

\lstinputlisting[title={/etc/asterisk/queues.conf},style=bash]{../conf/queues.conf}

La mayor parte de las opciones quedan comentadas, significativamente, hemos asociado tres agentes de la cola que hemos creado llamada ``Almacén'' con los respectivos nombres de usuario, y asociados a los tres dipositivos SIP de cada uno de ellos.

Por otro lado en el Plan de Marcación necesitamos hacer algunos ajustes a petición. Ajustamos un nuevo teléfono de tipo DDI ofrecido por la operadora:

\begin{lstlisting}[style=bash,title={/etc/asterisk/extensions.conf}]

;Numero General de UCA Autos
exten => 956001181,1,Goto(especiales,101,1)

\end{lstlisting}

Y por otro lado, creamos la extensión 101 en el contexto [especiales] para poder acceder a la cola \textbf{almacén} recién creada.

\begin{lstlisting}[style=bash,title={/etc/asterisk/extensions.conf}]

[especiales]

exten => 101,1,NoOp()
same => n,Playback(bienvenida)
same => n,Queue(almacen)
same => n,Hangup()

\end{lstlisting}

Finalmente queremos darle la opción a los usuarios del Call-Center para que se puedan poner en modo Pausa, por si tienen que atender otras necesidades.
Para ello vamos a crear Macros que se ajusten a la función de poner y quitar la Pausa del Agente dentro de la cola dentro del plan de marcación de las respectivas extensiones:

\begin{lstlisting}[style=bash,title={/etc/asterisk/extensions.conf}]

[macro-pausa]
exten => s,1,NoOp()
same => n,PauseQueueMember(almacen,SIP/${MACRO_EXTEN:1})
same => n,Hangup()

[macro-sinpausa]
exten => s,1,NoOp()
same => n,UnpauseQueueMember(almacen,SIP/${MACRO_EXTEN:1})
same => n,Hangup()

[extensiones]

exten => _73[2-4],1,Macro(pausa)
exten => _83[2-4],1,Macro(sinpausa)

\end{lstlisting}

Con esto quedaría el sistema base de Colas completamente funcional. Ahora por otro lado, queremos ofrecerle a UCA Autos eso que la mayor parte, ofrece como una solución con licencias bastante caras, el sistema de Estadísticas de control y Analisis.

Para ello vamos a utilizar la herramienta Asternic Stats, en su formato Open Source. Siguiendo el esquema de instalación ofrecido en la Wiki \footnote{WIKIAsterisk, Asternic Stats, http://wikiasterisk.com/index.php?title=Asternic\_Stats} \cite{website:asternic} tenemos montado todo en un tiempo relativamente corto, podemos ver en la Figura \ref{asternic} una captura de la lista de los tres usuarios de Almacén y su respectiva disponibilidad. 

\figura{estado_agentes.png}{scale=1}{Captura de la Pantalla de Disponibilidad de Agentes de Asternic}{asternic}{!ht}

En última instancia, quería preparar el sistema de Grabación de llamadas, y de escucha. Era importante hacer saber al Sr. Zutano que existían aspectos legales que tenía que tener controlados, con respecto a la grabación de llamadas (previo consentimiento del cliente, mediante una locución al inicio de la llamada, informando al respecto), para ello hacemos una pequeña modificación en la extensión 101, introduciendo esta locución específica por un lado.

Por otro lado, vamos a lanzar la aplicación de grabación Monitor, para que solo grabe cuando exista comunicación entre ambas partes de la comuniación. Aparte guardaremos los registros en ficheros individuales mezclados en formato wav, y tendran nombres distintos para poder identificarlos y que no se solapen unos con los otros:

\begin{lstlisting}[style=bash,title={/etc/asterisk/extensions.conf}]

exten => 101,1,NoOp()
same => n,Playback(bienvenida)
same => n,Playback(grabacion)
same => n,Set(GRABACION="grabacion-"${STRFTIME(${EPOCH},,%Y%m%d-%H%M%S)})
same => n,Monitor(wav,\${GRABACION},mb)
same => n,Queue(almacen)
same => n,Hangup()

\end{lstlisting}

Vamos a crear un enlace simbolico a nuestro directorio web raíz, a la carpeta de grabaciones de Asterisk para que puedan acceder a las mismas de forma directa el responsable. Asi podría acceder directamente desde el servidor web http://pfc-asterisk.local/monitor y ver todas las grabaciones:

\begin{lstlisting}[style=consola]
cd /var/www/
ln -s /var/spool/asterisk/monitor ./
\end{lstlisting}

Por otro lado, tenemos que informar por contrato a los empleados, que sus llamadas pueden ser escuchadas sin previa autorización, para ello remitimos al gerente a su responsable legal para que redacte el mismo. Con esto, habiendo allanado el terreno de lo legal, podemos proceder a realizar las modificaciones pertinentes

En cuanto a la posibilidad de ofrecer escucha, simplemente tenemos que añadir una nueva extensión para realizarla, en este caso lo haremos dentro del contexto [manager]

\begin{lstlisting}[style=bash,title={/etc/asterisk/extensions.conf}]

exten => 201,1,NoOp()
same => n,ChanSpy(SIP/3)
same => n,Hangup()

\end{lstlisting}

\subsection{Resolución: Colas en Asterisk}

En primer lugar, para comprobar el funcionamiento de las colas, es muy directo. Llamamos desde cualquier teléfono a la extensión especifica de las colas 101, y si tenemos teléfonos del rango 3X (extensiones de almacén), deberíamos de poder escuchar la llamada entrante. Si tenemos varios teléfonos para probar, podríamos comprobar como el primero pasa a estar en primera posición de la cola, y además todos los teléfonos asociados a la cola suenan a la vez por la estrategia \textbf{ringall}. Además los llamantes deberían recibir mensajes de audio indicándoles la posición en la cola, si todo esta correctamente parametrizado tal y como podemos ver en el desarrollo.

Podemos acceder simultáneamente, cuando hay varias llamadas de prueba cursándose en la cola al sistema Asternic Stats, y en el apartado Realtime tenemos que ver como los eventos van surgiendo de forma inmediata y automática. Además si todo el registro de la cola va realizándose correctamente, empezaremos a ver estadísticas en los otros apartados correspondientes, un ejemplo de estadísticas podríamos ver en la Figura \ref{estadisticas_asternic}.

\figura{estadisticas_asternic.png}{scale=1}{Ejemplo de Estadísticas Asternic Stats}{estadisticas_asternic}{!ht}

Para la escucha activa, cuando un agente o varios entre las extensiones 3X este hablando, podemos ejecutar desde un dispositivo la extensión 201, y automáticamente deberíamos poder estar escuchando la conversación en ese canal.

Finalmente para comprobar que las grabaciones se estan haciendo correctamente, simplemente tenemos que acceder a la URL http://pfc-asterisk.local/monitor para comprobar que van apareciendo ficheros de audio en su interior.