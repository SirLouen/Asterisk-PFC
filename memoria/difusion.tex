% -*-difusion.tex-*-
% Este fichero es parte de la plantilla LaTeX para
% la realización de Proyectos Final de Carrera, protejido
% bajo los términos de la licencia GFDL.
% Para más información, la licencia completa viene incluida en el
% fichero fdl-1.3.tex

\section*{Blog de Investigación: 10000 Horas}

El paso previo a la creación de WIKIAsterisk fue el proceso de generación de un Blog con fines de Investigación y Desarrollo con el sistema Asterisk. \cite{website:10000horas}. Una de las primeras inconveniencias que encontré, era la dificultad de entender el funcionamiento de un sistema de Telecomunicaciones y enfocarlo más como un sistema Informático. Pensé que si desarrollaba mi conocimiento paso a paso, podría resultar útil en un futuro a un recien llegado, o incluso en este caso, me serviría incluso de autoreferencia para apoyar este proyecto.

En la actualidad recibe aproximadamente unas 80-100 visitas diarias, curiosamente, incluso en Google Reader posee varios subscriptores, pese a no ser un blog demasiado activo.

Eventualmente, también me ha servido para recibir feedback por parte de la comunidad en mis desarrollos y así poder perfeccionarlos y depurarlos hasta conseguir niveles de utilidad aceptables. Además se han tratado temas que nunca fueron difundidos antes, y ha servido para sentar algunas bases y como punto de conocimiento para temas concretos.

La idea de ponerle 10000 Horas, fue por el hecho que existe un concepto que basa la idea, que se necesitan esas horas para conseguir la maestría en algo. Yo pretendía alcanzar la maestría en Asterisk pasando esas horas delante del sistema, y gracias al blog, mejorar la curva de aprendizaje de futuros lectores, para reducir la cantidad de tiempo a invertir drásticamente.




