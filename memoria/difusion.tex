% -*-difusion.tex-*-
% Este fichero es parte de la plantilla LaTeX para
% la realización de Proyectos Final de Carrera, protejido
% bajo los términos de la licencia GFDL.
% Para más información, la licencia completa viene incluida en el
% fichero fdl-1.3.tex

\section*{Blog de Investigación: 10000 Horas}

El paso previo a la creación de WIKIAsterisk fue el proceso de generación de un Blog con fines de Investigación y Desarrollo con el sistema Asterisk. \cite{website:10000horas}. Una de las primeras inconveniencias que encontré, era la dificultad de entender el funcionamiento de un sistema de Telecomunicaciones y enfocarlo más como un sistema Informático. Pensé que si desarrollaba mi conocimiento paso a paso, podría resultar útil en un futuro a un recién llegado, o incluso en este caso, me serviría incluso de autoreferencia para apoyar este proyecto.

En la actualidad recibe aproximadamente unas 80-100 visitas diarias como puede verse en la Figura \ref{estadisticas_10000horas}, curiosamente, incluso en Google Reader posee varios subscriptores, pese a no ser un Blog demasiado activo.

Eventualmente, también me ha servido para recibir feedback por parte de la comunidad en mis desarrollos y así poder perfeccionarlos y depurarlos hasta conseguir niveles de utilidad aceptables. Además se han tratado temas que nunca fueron difundidos antes, y ha servido para sentar algunas bases y como punto de conocimiento para temas concretos.

La idea de ponerle 10000 Horas, fue por el hecho que existe un concepto que basa la idea, que se necesitan esas horas para conseguir la maestría en algo. Yo pretendía alcanzar la maestría en Asterisk pasando esas horas delante del sistema, y gracias al blog, mejorar la curva de aprendizaje de futuros lectores, para reducir la cantidad de tiempo a invertir drásticamente.

\figura{estadisticas_10000horas.png}{scale=1}{Estadísticas de acceso del Blog 10000horas}{estadisticas_10000horas}{!ht}

\section*{Estadísticas de Acceso WikiAsterisk}

Para poder conocer las estadísticas de acceso del sistema MediaWiki conseguí dar con una aplicación complementaría llamada Firestats \cite{website:firestats} (Figura \ref{firestats} bastante completa y mucho más intuitiva y versátil que las propias extensiones de Mediawiki que podían agregarse (y la pobre interfaz de estadísticas que incluye de serie) \cite{website:wikifirestats}

Con esto podríamos en tiempo real, el impacto producido por el trabajo realizado a nivel de Difusión \ref{cap:difusion}.

La familiarización con herramientas nuevas, suele ser una tarea ardua y compleja según el momento en el que estamos tratando de lidiar con las mismas.
Uno de los puntos en el que más tiempo costó integrarme fue en el uso de las mismas, principalmente las enfocadas a la documentación, como fueron el manejo de sistemas WIKI basados en MediaWiki, y la documentación a través de Latex. 

\figura{firestats.png}{scale=1}{Estadísticas WIKIASterisk con Firestats}{firestats}{!ht}

\section*{Certificaciones Digium}

Durante el transcurso del desarrollo de la WIKI y finales de la primera fase de investigación, tuve la oportunidad de obtener la primera y más básica certificación de Digium, \textbf{Digium Certified Asterisk Administrator}, la cual está orientada a ofrecer a demostrar competencias básicas pero no esta ampliamente reconocida por el sector.

En segundo y último nivel en la actualidad, se encuentra el Digium Certified Asterisk Professional (Figura \ref{dcap}), el cual demuestra alta competencia en el manejo del sistema Asterisk, y está ámpliamente reconocida por todos los operadores de Telefonía a nivel mundial, más considerando que Asterisk esta siendo implantada no solo a niveles puros, sino incluso, existen referencia de sistemas Asterisk funcionando bajo el nombre de PBX de marcas reconocidas. Este auge podría suponer un gran interés especialmente en la comunidad orientada al mercado de las telecomunicaciones, de empezar a interesarse cada vez mas en Asterisk, y todo este proyecto, servir como medio para canalizar, ese interés especialmente dentro del entorno hispanohablante.

Una de las idea concebidas tras el surgimiento de la WIKI era enfocarla para cubrir todos los aspectos relevantes de la certificación dCAP y aprovechar la comunidad para dar cabida a toda la información  recursiva necesaria para ser una fuente de información lo suficientemente fidedigna para servir como punto de referencia.

\figura{dcap.png}{scale=0.5}{Logotipo de la Certificación dCAP}{dcap}{!ht}