% -*-planificacion.tex-*-
% Este fichero es parte de la plantilla LaTeX para
% la realización de Proyectos Final de Carrera, protejido
% bajo los términos de la licencia GFDL.
% Para más información, la licencia completa viene incluida en el
% fichero fdl-1.3.tex

\section{Planteamiento}

El desarrollo del proyecto se ha sustentado en tres pilares fundamentales:

\begin{enumerate}
	\item Desarrollo de una Wiki, para la base teórica que veremos más desarrollado en el capitulo 3
	\item Desarrollo de un Caso de Estudio aplicado en el capítulo 5, siguiendo el \textbf{Método del Caso}
	\item Creación de una Máquina Virtual, para montar un entorno de desarrollo y demostración basado en el Caso de Estudio, y visto en profundidad en el capítulo 4
\end{enumerate}

\subsection{Método del Caso}

El método del Caso es un sistema desarrollado para la enseñanza específicamente, y esta probado ser uno de los sistemas más eficientes de la actualidad para el aprendizaje teórico-práctico con un especial hincapié en el segundo elemento, y está más orientado al desarrollo que a la investigación. Es por esto, por lo que cada día en más programas de Postgrado se esta incorporando esta metodología satisfactóriamente.

Curiosamente, esta metodología es relativamente reciente, comparado quizá a otras formas centenarias o milenarias, desarrollada por la Universidad de Harvard\cite{website:hbsp} a principios del siglo XX \cite{mdc05}, orientada inicialmente a la carrera universitaria de Derecho, en la que se ponían casos reales de temas legales, y los alumnos tenían que enfrentarse a los mismos, siguiendo conceptos aprendidos en clase, y documentándose por sus propios medios para conseguir resoluciones positivas. Esto de alguna forma, incentiva el ingenio, dado que esta demostrado, que el sistema de enseñanza magistral aunque pueda intentar resultar más interactivo y participativo según la calidad del docente, la información realmente solo se produce de forma unidireccional del docente a los alumnos y el aprendizaje nunca suele derivarse fuera del material, dado que suelen ser sistemas con orientación dirigida.

Debido a que una de las motivaciones de este proyecto, era desarrollarlo siguiendo una metodología eminentemente práctica, y mi pretensión fuese, que transcendiera lo máximo posible, para personas interesadas en el sistema Asterisk en el futuro. Por ello la intención de encuadrarlo todo dentro de un marco didáctico y formativo, en vez de un marco de carácter divulgativo o investigativo.

Concrétamente para el desarrollo de este proyecto, el uso del Método del Caso, se dará aplicando un único Caso de Estudio, basado en una empresa del sector de la Automoción recién entrada al negocio, la cual desea introducirse ámpliamente, en las tecnologías de la comunicación de forma eficiente, y para ello toma la colaboración de nosotros, que representamos una pequeña empresa que implanta sistemas de comunicaciones basados en Asterisk. Todo esto queda desarrollado al detalle a lo largo del \textbf{Capítulo 5}.

Dentro del capítulo en cuestión, cada sección se va subdividiendo en varios subapartados. La idea conceptual es la siguiente:


\begin{enumerate}
	\item Primero se enuncia el caso de estudio, explicando los problemas, las necesidades que surgen para cubrir ese problema, y algunas ideas puntuales, sobre como orientar la solución, pero de manera solo lo suficiéntemente específica para poder ajustarla de manera efectiva.
	\item En segundo nivel, se trabaja el caso desde una de las múltiples perspectivas posibles de resolución que existe, aportando al lector, un grupo de alumnos que comparten el mismo interés, o en grupo con un docente que dirija la sesión, una guía para orientar el camino que da lugar a una posible solución concreta
	\item En tercer lugar, da lugar al sistema de comprobación de que la solución llevada a cabo original, o alternativa, produce los efectos deseados según la especificación del Caso de Estudio en cuestión.
\end{enumerate}

En resumen, puede verse un la Figura \ref{estructura_mdc} de ejemplo que muestra las diferentes fases de un Caso siguiendo la Metodología del Caso en cuestión.

\figura{estructura_mdc.png}{scale=1}{Esquema general de una sesión siguiendo el Método del Caso}{estructura_mdc}{!ht}

\section{Planificación}

Para considerar la planificación del Proyecto se han podido observar dos periodos comprendidos antes del inicio, y al comienzo de la escritura de este proyecto.

\subsection{Aprendizaje e Investigación}

En primer lugar se dio una fase de aprendizaje del sistema Asterisk. Esta fase comenzó a mediados del 2010, durante 6 meses, acompañado de un curso de la empresa colaboradora y que actualmente lidera el proyecto Asterisk, Digium \cite{website:digium}. Este curso era de carácter inicial (llamado \textbf{Fast Start}), y cubría aspectos básicos de inicio para empezar a realizar algunos desarrollos de baja escala. Podría considerarse un curso de iniciación al mundo de las PBX orientado específicamente a Asterisk. 

En segundo lugar, habiendo pasado varios meses tras una larga parada en el proceso, que fue a finales de 2011, a raíz de un curso avanzado ofrecido por la misma compañía, Digium, acerca de Asterisk en profundidad, se dio el impulso a un segundo aprendizaje e investigación sobre la materia aún más completo, y orientado al entorno de la ingeniería del software sobre la que se sustentaba Asterisk. Esta segunda fase se dio por otro periodo de unos 6 meses, hasta mediados de Marzo de 2012. Durante esta segunda fase, es cuando contacto con el tutor del Proyecto para formalizar la idea, y empezar a sentar las bases.

La programación general en estos meses quedaría definida según la Figura \ref{primera_fase}, todas las figuras de planificación han sido diseñadas con la herramienta Planner \cite{website:planner}

\figura{primera_fase.png}{scale=1}{Planificación general del Proyecto}{primera_fase}{!ht}

\subsection{Desarrollo e Innovación}

A principios de Marzo 2012, se diseña la programación para la creación de este proyecto que durará hasta finales de Junio según previsto, lo que incluye solo la fase de desarrollo. Esta programación se divide en los tres aspectos principales en los que se basa este proyecto: El Caso de Estudio, la Máquina Virtual de Desarrollo y la WIKI teórica.

La idea, era empezar a desarrollar conceptos teóricos, aglutinando toda la información recopilada en la Fase de Investigación, y condensándolo en la WIKI en forma de artículos organizados. Esta fase tenía la previsión de durar, los dos primeros meses.

De forma concurrente, durante el desarrollo de varios Artículos más significativos, se iría desarrollando el Caso de Estudio haciendo referencia a los mismos. Esto empezaría aproximadamente dos semanas después del inicio del desarrollo de la WIKI, y acabaría unas dos semanas después justo coincidiendo con la fecha límite del proyecto.

Y también de forma simultánea, el Caso de Estudio se sustentaría en en desarrollo de la máquina virtual, que comenzaría prácticamente también a la vez, y acabaría una semana antes aproximadamente, dado que la última semana serviría adicionálmente para configurar y retocar en términos generales los detalles de este documento.

Todo esto quedaría reflejado en la Figura \ref{segunda_fase}.

\figura{segunda_fase.png}{scale=1}{Planificación del Desarrollo del Proyecto}{segunda_fase}{!ht}

\subsection{Fases de Creación}

Durante el proceso de desarrollo, las distintas secciones se programaron de forma equitativa, agrupándolas por módulos conceptuales. El procedimiento era el siguiente:

\begin{enumerate}
	
	\setlength{\itemsep}{10pt}

	\item Desarrollo de la teoría y las bases que sustentaban la sección
	\item Pruebas prácticas en Máquina Virtual, desarrollo, puesta a punto y documentación
	\item Diseño y creación del Caso de Estudio que encuadre el problema planteado
\end{enumerate}

El desarrollo de las secciones puede contemplarse en la Figura \ref{tercera_fase} y la descripción de las mismas fue la siguiente:

\figura{tercera_fase.png}{scale=1}{Fases del Desarrollo del Proyecto}{tercera_fase}{!ht}

\begin{enumerate}

\setlength{\itemsep}{10pt}

	\item \textbf{Introducción} 
	
	En este apartado, se pretende contemplar una visión panorámica del sistema Asterisk, incluyendo todos los datos generales sobre el sistema, historia y su funcionamiento interno, Arquitectura en esencia, los procedimientos de Instalación y Actualización 
		
	\item \textbf{Protocolos basados en la Red} 
	
	La idea general, es dar pinceladas muy generales sobre los protocolos más distendidos, pero no concretar demasiado, dado que la intención de este proyecto no es profundizar tanto en la Telefonía IP, sino en el sistema Asterisk en concreto y como maneja las situaciones a nivel productivo. Es por ello que que el planteamiento a seguir es el de desarrollar información acerca de estos protocolos enfocándolos exclusivamente a su papel que desempeñan dentro del sistema Asterisk
	
	\item \textbf{Primeros Pasos en el Plan de Marcación} 
	
	Se pretende agrupar, todo el procedimiento general para empezar a desarrollar los primeros planes de marcación, todos las aplicaciones más comunes y una primera aproximación para el lector. En esto se incluyen no solo todas las ideas generales, basadas en este sistema de programación propio de Asterisk, con carácter de script, sino sus referencias más comunes. Además era importante tener una aproximación inicial al los protocolos dado que la aplicación de un plan de marcación siempre va directamente relacionado a los mismos.
	
	\item \textbf{Grupos de Módulos} 
		
	La intención es ofrecer una visión general sobre los Módulos que fundamentan los pilares básicos de Asterisk, agrupados por categorías o en función del papel que desempeñan a nivel empresarial, o funcional. Evidentemente es la parte más densa del proyecto, aunque mas o menos se dedica el mismo tiempo por módulo dado que aproximadamente es posible desarrollar la misma cantidad de contenido para los respectivos.\\
	
	\begin{itemize}
	 
	  \setlength{\itemsep}{10pt}
	
		\item Sistema de Buzones de Voz
		\item Call Centers y Colas de Llamadas
		\item Grabación y Monitorización de Llamadas
		\item Sistemas de Gestión de FAX
		\item Bases de Datos y Configuración en RealTime
		\item Generación Automática de Llamadas y Marcadores
		\item Registro de Llamadas y Eventos
		\item Sistemas de Conferencia
		
	\end{itemize}
	
	Además de todo esto también cubrir ciertas Interfaces Web Gráficas que hacen alguno de estos módulos más versátiles y dinámicos por definición
	
	\item \textbf{Introducción a las Interfaces de Telefonía Analógica} 
	
	Dado que la pretensión de este proyecto es focalizar principalmente en los sistemas de voz sobre IP, también existe un nicho en Asterisk relacionado al mundo de la telefonía analógica y es necesario cubrirlo para aportar más soluciones y más considerando que en gran medida el tejido empresarial todavía se nutre de los mismos. En esta parte la idea es cubrir tanto las tarjetas que incorporan interfaces de telefonía analógica como los Gateway que interfasan a través de un protocolo SIP con nuestro Asterisk.
	
	\item \textbf{Interfaces de Comunicaciones Unificadas} 
	
	La idea detrás de esto es poder demostrar el verdadero ``poder'' del sistema Asterisk, dado que el manejo de las interfaces que ofrece Asterisk es lo que permite las posibilidades ilimitadas de interconexión con otros sistemas y ofrecer las sinergias relativas. En este apartado la idea es detenerse un poco más y da lugar a las pruebas, mayor investigación a fondo y desarrollos más precisos. Principalmente hay dos áreas de cobertura en función de cada una de las dos interfaces de Asterisk:\\
	
	\begin{itemize}
	  \setlength{\itemsep}{10pt}
		\item Asterisk Manager Interface
		\item Asterisk Gateway Interface 
	\end{itemize}

\item \textbf{Creación de Mecanismos de Respuesta Interactiva de Voz} 

Finalmente, aunando todo lo visto anteriormente, poder desarrollar y profundizar en el contexto, dar a conocer las técnicas de Interacción por Voz entre las máquinas y las personas, y ofrecer el concepto diseñar un sistema de Telefonía 2.0 de pequeña escala, para demostrar las bondades de esta tecnología en desarrollo

\end{enumerate}
