% -*-conclusiones.tex-*-
% Este fichero es parte de la plantilla LaTeX para
% la realización de Proyectos Final de Carrera, protejido
% bajo los términos de la licencia GFDL.
% Para más información, la licencia completa viene incluida en el
% fichero fdl-1.3.tex

\section{Conclusiones}

Tras mostrar a la comunidad de Asterisk el proyecto, y recibir una amplia aceptación ha resultado una gratificación en términos generales. Desde el inicio del proyecto de investigación y la puesta en marcha del Blog de Difusión 10000 Horas \ref{cap:difusion} ya podían observarse múltiples atisbos de interés. Las estadísticas hablan hoy y tengo firme convicción que el proyecto no ha hecho más que empezar.

Poder realizar un primer proyecto a nivel formal que integra múltiples mecanismos de desarrollo y documentación ha sido una idea interesante planteada por el tutor Manuel Palomo combinada con mis conocimientos sobre Asterisk.

\subsection{Sobre WIKIAsterisk}

La estructura esta diseñada, para servir como una guía práctica para el lector, y que dado el caso, pudiera seguirse como un manual de referencia. Existen guías y documentos en formato papel, y no se descarta en un futuro poder editar la misma si así resultase en una difusión aún más distendida.

Podríamos considerar que el desarrollo sigue un camino, iniciando al lector en los aspectos generales, pasando por aspectos prácticos como la instalación, y conceptuales generales, como el aprendizaje de la generación y creación de un plan de marcación, desde niveles básicos a niveles más sofisticados. A partir de ahí se vuelcan todas las herramientas en orden de importancia relativa en forma de artículos y bajo mi visión personal puede servir como traza para el lector pueda ir tomando los aspectos que más le interesen en función de sus necesidades.

El aspecto visual me resulta impecable y a diferencia de un libro, el cual suele estar basado en un índice, y recorrer las páginas, aun intuitivo puede resultar engorroso, todo el sistema referencial que aporta un sistema WIKI aporta una potencia que hasta la fecha, no pensaba que pudiera resultar tan impactante. 

Particularmente tengo que reconocer que con diferencia, lo que más me ha sorprendido ha sido la facilidad con la que podremos crear referencias de un punto a otro de la WIKI, y la facilidad con la que podemos entrelazar en consecuencia los artículos. Un caso curioso, podría decir, que el Artículo que habla sobre SIP es de lejos, el artículo más referenciado en toda la WIKI dado que como explicaba en la Introducción, realmente es el protocolo de Telefonía IP más distendido en la actualidad y prácticamente todos los aspectos de Asterisk tienen algo que ver con el mismo.

\subsection{Método del Caso y Caso de Estudio}

Una de las cosas que tenía clara desde el primer momento, era diseñar una metodología de explicación de los contenidos prácticos de una forma que fuera lo más apartada de lo convencional posible, que hasta la fecha, me había resultado algo escabrosa especialmente en el despliegue relacionado a sistemas informáticos.

Además ha servido para mi por primera vez, en aplicarlo de manera ``formal'' y quizá me plantearía en un futuro una prueba piloto para aplicarlo con carácter docente extraoficialmente. Me ha sorprendido gratamente, que habiendo utilizado un caso, basado en una mediana empresa dentro del sector de la automoción a la cual me he sentido fuertemente asociado hasta una fecha relativamente cercana, se me han ido ocurriendo problemas de una manera prácticamente espontánea, y me veía aprovechando mis ratos libres, para idear de manera ingeniosa, como podía desarrollar un planteamiento práctico, en un supuesto caso real de implantación, para que resultara lo más creíble y realista posible.

En términos generales, puedo afirmar ahora que el caso ha quedado bastante completo y cubre todos los aspectos generales como para poder integrar a un usuario en Asterisk, enfocándolo en el marco conceptual de las Comunicaciones Unificadas.

La prueba de fuego puede que realmente se encuentre, en medir con terceras personas si realmente este sistema pudiera ser lo suficientemente didáctico para implementarse como una manera dinámica de implantar una metodología de enseñanza en las formaciones relativas a Asterisk y otros sistemas semejantes.

\subsection{Máquina Virtual}

Considerando las limitaciones de Hardware tuve que plantearme como desplegar un caso práctico completo, y funcional para demostrarme a mi mismo que era posible integrar, probar y desarrollar toda la funcionalidad de Asterisk a pequeña escala en mi domicilio particular. Al principio, pensaba que con una sola máquina virtual iba a poder cubrir toda la práctica, pero al poco tiempo surgió la necesidad de crear una segunda máquina virtual, y que afortunadamente, podía convivir sin problemas de rendimiento dentro del mismo servidor, y a la necesitábamos considerar como ``Proveedor de telefonía IP'' o ITSP (Internet Telephony Service Provider). 

Con esto y en esencia, todo quedaba cubierto y la máquina del Proveedor, quedaría como una ``Caja Negra'' para demostrar como la interoperabilidad con el mismo, no era dependiente a excepción de los parámetros de configuración que un supuesto Proveedor de Telefonía IP nos proporcionase con el contrato.

Este tipo de configuraciones son muy comunes en los cursos de Asterisk, donde el ordenador del docente, sirve como ``Proveedor de Telefonía'' y los equipos de alumnos son las máquinas Asterisk que operan con el mismo.

En general este planteamiento ya lo tenía tan asimilado desde el primer momento, que no me resulto demasiado complejo, la implementación de sucesivas mejoras para adaptar las necesidades del proyecto a cada momento presente en la línea del tiempo.

\section{Dificultades Técnicas Superadas}

La familiarización con herramientas nuevas, suele ser una tarea ardua y compleja según el momento en el que estamos tratando de lidiar ellas.

Debo reconocer que uno de los aspectos que más tiempo me costó integrar fue en el uso de las mismas, principalmente las enfocadas a la documentación, como fueron el manejo de sistemas WIKI basados en MediaWiki, y la documentación a través de \LaTeX. 

Creo que esto suponía más un problema de falta de adecuación, y conceptual, en el sentido que a priori no entendía los principios básicos de estas herramientas considerando que estaba acostumbrado a utilizar otros medios más distendidos de documentación (en general hablando de sistemas WYSIWYG que a pesar que de mayor o menor medida).

Pasado el primer mes de uso intensivo de estas herramientas pude observar como mis dificultades fueron quedado atrás, y acepté que realmente para escribir un documento no era necesario que resultara estéticamente vistoso en tiempo real, algo que hasta la fecha era totalmente ajeno a mi experiencia.

Por otro lado en cuanto a lo que el desarrollo del proyecto hace referencia debo reconocer, que a pesar que me costó tiempo entender el funcionamiento del sistema Asterisk, gracias a la formación específica, y el tiempo que permanecí en la primera fase en la que se desarrollo todo el proceso de aprendizaje e investigación en la materia, facilitó de una gran manera el poder diseñar el grueso del proyecto técnico en un intervalo limitado de tiempo razonable.

Afortunadamente, el uso de lenguajes de programación aplicados en la materia, como PHP \cite{website:php} \cite{gilmore10}, durante el transcurso del desarrollo, ha sido uno de mis puntos fuertes, lo que no me supuso mayor dificultad, que aprender a integrar este conocimiento dentro de las posibilidades que me ofrecía Asterisk a nivel de Interfaz.

Finalmente, también debo destacar que al final, el hecho de haber conseguido integrar todas las herramientas aquí utilizadas, me ha servido para dar un paso adelante en varios niveles, y en la actualidad ya me esta sirviendo para otros ámbitos foráneos al presente proyecto, quizá resultado de la ventaja de incorporar tecnologías eficientes en nuestro conocimiento.

\section{Planteamientos Futuros}

Durante la creación de este proyecto, iban surgiendo ideas nuevas, sobre posibles planteamientos que pudieran llevarse a corto, medio y largo plazo. De forma generalizada, podría decirse que la mayor parte de los planteamientos que surgieron a corto, decidieron llevarse adelante, aún no estando previstos dentro de la idea original, y exceptuando contados casos, tuvieron que dejarse a un lado. La mayor parte de estos conceptos a corto, están fundamentados en determinadas secciones de la WIKI, que aún comentadas por encima, deberían quedar más detalladas en profundidad, dado la repercusión que pueden suponer en ofrecer el máximo de contenidos de calidad. En realidad quizá haya sido por cuestiones de previsiones de tiempo que alguna instancia se hubiera quedado en el tintero.

A medio Plazo, mis ideas se enfocan principalmente en un tema que surge levemente en varias ubicaciones de este Proyecto, y tiene que ver con una plataforma de relativa reciente creación, llamada Asterisk Communications Framework (Asterisk SCF), y esta basada en la idea de poder desplegar una arquitectura distribuida, que ofrezca la flexibilidad semejante al P2P de Skype, pero basada exclusivamente en sistemas de Servidores escalables. Esto esta más orientado a Operadoras, y al nivel Carrier-Grade \cite{website:carriergrade} ya que se observó con el tiempo, que Asterisk podía convertirse en un cuello de botella, dada su baja escalabilidad y era frecuente recurrir a SIP Proxys \cite{goncalves10} fuera de Asterisk específicos que si ofrecían estas posibilidades de escalabilidad de forma nativa como Kamailio \cite{website:kamailio}

A largo Plazo, mi pretensión se encuentra focalizada, en poder desarrollar una WIKI adjunta a WIKIAsterisk.com, pero exclusivamente basada en el entorno de desarrollo de Asterisk, para ofrecer una introducción y profundización paulatina, y poder extender la comunidad de desarrollo a más ámbitos. Además al igual que la WIKI original está originalmente orientada a mostrar todo el contenido disponible para prepararse cara a la certificación \textbf{Digium Certified Asterisk Professional} (dCAP) según noticias de Digium, plantean lanzar una certificación \textbf{Digium Certified Asterisk Developer}, que irá enfocada a los desarrolladores del mismo.

Además como se comenta en el capitulo de Difusión \ref{cap:difusion}, otro de los aspectos obligados, es la creación de comunidad, dado que la verdadera eficiencia de un sistema de este tipo es difícilmente manejable por una única persona, dado que los sistemas se actualizan regularmente, van surgiendo novedades en varios niveles y conceptos, y esto repercute en una ``desactualización'' general, que acaba degenerando en la obsolescencia de la creación. A no ser que en este caso, yo personalmente, me encontrara exclusivamente enfocado en el entorno de Asterisk, cosa que resultase poco probable, lo más típico sería que pequeñas aportaciones de mucha gente fueran similares, a una gran aportación de una sola persona como es este proyecto.

Finalmente uno de los aspectos más interesantes que que también han quedado abiertos, es el desarrollo de Asterisk para otras plataformas diferentes a Linux, y concrétamente a Ubuntu Server/Debian, al cual ha estado muy enfocado el proyecto en general. Uno de los grandes atributos de Asterisk es que se integra en todo tipo de sistemas basados en la plataforma UNIX perféctamente.