% -*-caso_de_uso_1.tex-*-
% Este fichero es parte de la plantilla LaTeX para
% la realización de Proyectos Final de Carrera, protejido
% bajo los términos de la licencia GFDL.
% Para más información, la licencia completa viene incluida en el
% fichero fdl-1.3.tex

Con todo el sistema, preparado para empezar a trabajar con Asterisk, a continuación presento un ejemplo como caso de uso para una empresa ficticia, la cual tiene un enorme interés en implementar un sistema Asterisk, dadas las características que este le aportara para su negocio.

\color[rgb]{0,0,1}

\section{Introducción}

Don Zutano Doe, gerente, y propietario de un nuevo concesionario de automóviles llamado UCA Autos, basado en una prestigiosa marca a nivel nacional, es un gran aficionado a todos los adelantos tecnológicos y piensa que estos influirán en gran medida en el amplio desarrollo de su negocio. A través de un compañero de su club de tenis, que trabaja en una empresa de telefonía, le ha dado a conocer un nuevo sistema de comunicaciones que le podría aportar una serie de funcionalidades que le resultaron bastante interesantes para implementar en su nuevo negocio. Gracias a este contacto, tuvo la oportunidad de establecer relaciones con una modesta empresa local, con cierta experiencia en este ámbito.

En la primera reunión con el responsable de proyectos, el Sr. Doe pudo trasmitir sus pretensiones iniciales. Dada que la estructura de la empresa aun estaba por ser determinada, de momento solo necesitaba funcionalidades básicas para cubrir los aspectos fundamentales de servicios primarios a nivel de comunicaciones para el negocio.

A priori la estructura departamental estaba definida, y había un listado de personal para poder definir una primera instantánea de cómo debería funcionar todo.

El responsable de proyecto pudo describirlo de la siguiente forma.

Existían 6 departamentos:

\begin{enumerate}

\item Departamento Administración
\item Departamento Comercial
\item Departamento de Almacén y Logística
\item Departamento de Postventa y Servicio
\item Departamento de Marketing y Calidad

\end{enumerate}

Cada departamento tenía un responsable, y todos ellos dependían directamente, del Sr. Doe. Dentro de cada departamento existía un número variable de operarios, administrativos, comercial y personal en general que dependían de cada responsable.

Cada departamento debía tener un número de teléfono de contacto, y además, existía un teléfono general cara al publico. En total 6 números de teléfono directos.

La empresa todavía no era muy grande, y disponía de un número limitado de personal de los cuales los que realmente necesitaban acceso telefónico, se distribuían de la siguiente forma:

Departamento de Administración: Responsable + 3 administrativos
Departamento Comercial: Responsable + 4 comerciales
Departamento de Almacén: Responsable + 3 operarios
Departamento de Posventa: Responsable + 2 recepcionistas
Departamento de Marketing: Responsable + 2 operadores.
Más el Gerente

En total 20 usuarios iniciálmente

De momento, el Sr. Doe no tenía demasiado clara una posible estructura jerárquica con restricciones de llamadas, y relaciones entre departamentos, así que la primera idea, era tener un sistema de telefonía simple, en el que todos pudieran llamar y recibir llamadas abiertamente, pero no descartaba en un futuro, implementar esas restricciones, e incluso tener algún sistema de control.

\newpage

\color[rgb]{0,0,0}

\subsection{Instalación del sistema Asterisk}

Dado que el nivel Hardware es algo excesivamente cambiante, voy a obviar las características técnicas del entorno y del servidor, haciendo referencia al apartado de elección de servidor para los aspectos determinantes a la hora de elegir una u otra preferencia.

La instalación del sistema Asterisk, se realizara sobre la versión actual mas estable, la versión 1.8, momento en que escribo esta información. He de determinar, que en la actualidad aun existen múltiples ramas dentro de la elección de versionado para el sistema Asterisk. Principalmente existen 5 fundamentales:

\begin{enumerate}

\item Rama extremadamente conservadora, y un Fork de Asterisk del sistema base llamado Asterisk-RSP (Real Solid Patchset), que se basa en la teoría de conservarse en un sistema Asterisk totalmente obsoleto a nivel de funcionalidades emergentes (siempre considerando que Asterisk es una solución de comunicaciones, y no te telefonía exclusivamente). Esta fundamentada en la versión 1.4, versión que decidieron era el momento de parar la vorágine de avances, y centrarse en la resolución de deficiencias de seguridad del sistema
\item Rama por necesidad o moderadamente conservadora, basada en la anterior versión de Asterisk, 1.6.2. Realmente en esta situación se encuentran instalaciones de Asterisk con desarrollos a medida que no pueden actualizarse porque el sistema se volvería inestable. Realmente todo sistema Asterisk puro, es recomendable actualizarlo dado que la versión 1.6.2 tiene un soporte limitado.
\item Basada el progreso de Asterisk, en este caso todas las instalaciones basadas en la versión 1.8 la cual trae bastantes mejoras, y simplificaciones a nivel de configuración. Se considera por Digium, la versión estable del sistema
\item Basada en el progreso extremado, instalaciones basadas en Asterisk 1.10 (o como le llaman ahora, Asterisk 10). Realmente son versiones de prueba, y jamás recomendadas para instalaciones en entornos de producción como viene siendo habitual en la mayoría de los entornos de administración de sistemas.
\item Liberada recientemente a la comunidad, una rama similiar a Asterisk-RSP pero mantenida por Digium, y basada en las versiones mas recientes del sistema que se encuentran en fase estable. Podría considerarse una versión de Soporte a Largo Plazo, y es denominada Asterisk-Certified. En estos momentos esta empezando a imponerse entre la mayoría de los usuarios profesionales que venían siguiendo por sistema la opción número 3.

\end{enumerate}

Para nuestro cliente, en este caso, vamos a seguir la opción número 5, ya que se trata de un servidor de producción y queremos ofrecerle un nivel 2 de servicio según el acuerdo en la capa de servicio estandarizado \footnote{Wikipedia, Soporte Técnico. http://es.wikipedia.org/wiki/Soporte\_t\%C3\%A9cnico} \cite{walker01}

Para el momento, la versión Certificada mas reciente es la versión 1.8.11-cert1 asi que descargaremos las fuentes de la siguiente URL:

\href{http://downloads.asterisk.org/pub/telephony/certified-asterisk/releases/certified-asterisk-1.8.11-cert1.tar.gz}{Descarga Asterisk Certified 1.8.11-cert}

Y seguiremos el método de instalación descrito en el articulo Wiki asociado correspondiente.

Una vez con todo el sistema Asterisk desplegado, nos detenemos a comprobar el consumo de recursos en ese momento determinado para comprobar que podremos realizar el resto del despliegue sin mayor inconveniencia.

\figura{capacidad_tras_instalacion.png}{scale=1}{Capacidad del disco duro tras la instalación completa de Asterisk}{capacidad_tras_instalacion}{!ht}

Todavía tenemos suficiente espacio para añadir componentes de menor magnitud. Existen dos componentes adicionales que cumplirán funciones locales específicas y se relacionarán con nuestra máquina Asterisk prácticamente desde los inicios de su configuración. 

Se trata del paquete LAMP (Linux + Apache como servidor Web + MySQL como SGBD de nuestra BD + PHP como lenguaje de programación en entorno web que se conjugará con nuestro servidor web para poder ofrecer aplicaciones combinables con Asterisk como veremos mas adelante) y por otro lado algún servidor de Correo como Postfix. Para ello nos valemos de la herramienta \negrita{tasksel} de Ubuntu que realiza la instalación completa de la forma más ágil para nuestro propósito. Con estas instalaciones apenas consumimos poco más de 100MB así que seguimos teniendo suficiente espacio para continuar el proceso tranquilamente.

\subsection{Comprobación: Instalación del sistema Asterisk}

Para comprobar que el sistema ha sido instalado y configurado correctamente, simplemente podríamos probar a acceder a la Interfaz en Línea de Comandos de Asterisk:

\begin{lstlisting}[language=sh]
# asterisk -r
\end{lstlisting}

Y a continuación comprobar que todo funciona bien lanzando un comando CLI para recargar toda la configuración:

\begin{lstlisting}[language=sh]
CLI> core reload
\end{lstlisting}

En caso que no aparezcan errores significa que todo esta bien. Si no hemos emplazado todos los ficheros de configuración tales como \textbf{asterisk.conf} o \textbf{indications.conf} es probable que salga algún error, por falta de los mismos. Para ello deberemos volver a repasar la guía de Instalación y Configuración inicial provista en la WIKI de Asterisk.

\newpage

\subsection{Configurando equipos y el entorno SIP}

Ahora daríamos paso a la configuración en primera instancia de los requerimientos solicitados por el Sr. Doe en función de la estructura departamental.

Vamos a considerar, que todavía el presupuesto es demasiado ajustado para la empresa del Sr. Doe como para andar comprando teléfonos que operen con el protocolo SIP \cite{johnston09} (conocidos popularmente como Teléfonos VoIP). Pero hemos comprado una remesa de Auriculares con micrófono integrado para cada una de las extensiones a configurar, que serán adaptados al cada PC de cada usuario y funcionales, utilizando un software capaz de trasmitir audio e inicializar sesión en nuestra máquina Asterisk utilizando el protocolo SIP.

Creamos en primer lugar el fichero de configuración dentro de /etc/asterisk/ llamado sip.conf donde añadiremos la configuración especifica de cada una de las 20 extensiones solicitadas por el gerente:

\begin{lstlisting}[language=bash,title={/etc/asterisk/sip.conf}]

[general]
; Nuestro idioma
language = es
; Cambiamos el puerto para ofrecer seguridad por ocultacion
bindport = 35060
allowguest = no

\end{lstlisting}

En primer lugar para la creación de este fichero, hemos seguido las reglas básicas de configuración para el protocolo SIP como podemos ver en el correspondiente articulo de la Wiki. \footnote{SIP. http://wikiasterisk.com/index.php?title=SIP}. Significativamente podemos observar que la estructura que hemos seguido para denominar las extensiones, aun no siendo demasiado recomendada, puede verse en la figura \ref{tabla_extensiones}

\figura{tabla_extensiones.png}{scale=1}{Estructura Inicial de Extensiones de Usuario}{tabla_extensiones}{!ht}

Las extensiones constarán de dos dígitos. El primero se encarga de definir el departamento, y el segundo, el puesto. La extensión 10, como extensión especial para definir concrétamente la del Sr. Doe, el resto, si el número para designar el puesto es el número 1, correspondería al Responsable de Sección, y los sucesivos dígitos, corresponden al resto de los empleados. Además hemos creado cuatro plantillas para un cometido concreto: 

\begin{lstlisting}[language=bash,title={/etc/asterisk/sip.conf}]

[telefonos](!)
; Vamos a impedir el Transcoding utilizando solo el codec Alaw
disallow = all
allow = alaw
dtmfmode = rfc2833
host = dynamic
qualify = yes
type = friend

[gerencia](!)
context = gerencia

[manager](!)
context = manager

[resto](!)
context = extensiones

\end{lstlisting}

La primera, teléfonos, es una plantilla que engloba todos los parámetros comunes a las extensiones de la centralita. En este caso, nos interesa trabajar en exclusiva con el códec Alaw (G.711 EU), dado que tenemos una conexión bastante buena, pero nuestra máquina no es demasiado potente, además no queremos invertir en códecs licenciados.

Las tres siguientes plantillas, hacen referencia al "nivel" de los usuarios, y en función de este, los mandaremos a un contexto u otro del Dialplan que construiremos a continuación. A priori, el señor Zutano no nos hizo referencia concreta, sobre la posibilidad de que unos usuarios pudiera hacer cosas que otros no, pero anticipándonos por nuestra experiencia, empezamos a "habilitar" esta funcionalidad.

El sistema de contraseñas, teniendo a lo más básico, aun siendo una mala práctica, hemos utilizado la contraseña por defecto 1234 para todas las extensiones, pero al menos encubierto bajo contraseñas de algo mas de seguridad con un patrón MD5, para mostrar algo más de compromiso.

Finalmente y volviéndonos a adelantar a una probable futura necesidad, hemos asociado a cada extensión su correspondiente buzón de voz, que configuraremos a voluntad si surge el caso más adelante.

Pongo un ejemplo de varias extensiones "que cumplan estas condiciones":

\begin{lstlisting}[language=bash,title={/etc/asterisk/sip.conf}]

; Gerente
[10](telefonos,gerencia)
md5secret=2b45660a7b1155943f0132f05bd0e34d
mailbox = 0@admin

; Responsable Administracion
[11](telefonos,manager)
md5secret=6acb3eb4a4939b22ea6709e87ec79311
mailbox = 1@admin

[12](telefonos,resto)
md5secret=bb24a2de8a36fdd14e358493cd5abac5
mailbox = 2@admin

\end{lstlisting}

Ahora pasamos a comprobar que todo ha ido bien. Accedemos a la CLI de Asterisk y con el comando:

\begin{lstlisting}[language=sh]
CLI> sip reload
\end{lstlisting}

Vemos como nuestra configuración carga adecuadamente. Ahora vamos a comprobar que los pares han sido configurados correctamente. Observando el despliegue de sistemas de la empresa UCA Autos, comprobamos que todos los puestos trabajan en un sistema operativo de Microsoft. En este caso decidimos que vamos a utilizar un teléfono por software, dado que el Sr. Doe de momento no se ha decidido en el modelo de teléfono físico que quiere comprar. El Softphone que hemos elegido por su reconocida estabilidad y calidad es uno gratuito, pero no libre, llamado X-Lite. Tenemos constancia de múltiples Softphone Open Source, pero bajo este sistema operativo, no tienen demasiada estabilidad y pueden complicarnos la vida.

\subsection{Comprobación: Configurando equipos y el entorno SIP}

Configurando la extensión del Sr.Doe de ejemplo como podemos ver en la siguiente Figura \ref{config_xlite}, captura de pantalla de X-Lite.

\figura{config_xlite.png}{scale=1}{Configuración de una Extensión con el Programa X-Lite v4.1}{config_xlite}{!ht}

Comprobamos que se registra correctamente:

\begin{lstlisting}[style=consola]
-- Registered SIP '10' at 192.168.1.200:16992
CLI> sip show peers
Name/username Host           Dyn Forcerport ACL Port   Status
10/10         192.168.1.200  D              N   16992  OK (3 ms)
\end{lstlisting}

Así que ya podemos proceder a registrar las 20 extensiones, en los 20 equipos de nuestros usuarios y ya estarían listos para empezar a operar con nuestra primera instancia de sistema Asterisk.

\newpage

\subsection{Configuración del Primer Plan de Marcación}

Ahora los teléfonos no pueden hacer casi nada, excepto llamar a una extensión si saber la ruta exacta, y sin pasar por nuestro sistema Asterisk. Nos interesa que por lo menos puedan realizar llamadas entre ellos marcando el número de extensión según la estructura que definimos anteriormente. Para ello vamos a crear el fichero de configuración del Dial Plan según veremos a continuación:

\begin{lstlisting}[language=bash,title={/etc/asterisk/extensions.conf}]
; --- Plan General de Extensiones

[extensiones]

exten => 10,1,NoOp()
same => n,Dial(SIP/10)

exten => 11,1,NoOp()
same => n,Dial(SIP/11)

exten => 12,1,NoOp()
same => n,Dial(SIP/12)

\end{lstlisting}

En primer lugar, creamos números de marcación específicos para todas las extensinoes, he puesto las extensiones del ejemplo para continuar con la misma línea

\begin{lstlisting}[language=bash,title={/etc/asterisk/extensions.conf}]

; --- Contexto especifico de los Managers

[manager]

include => extensiones

\end{lstlisting}

Dado que de momento no existe nada especial para los Mánager, definimos un contexto que apunte al resto de las extensiones, como si fueran extensiones normales

\begin{lstlisting}[language=bash,title={/etc/asterisk/extensions.conf}]

; --- Contexto especifico del Gerente

[gerencia]

include => manager

; --- Otros Sistemas Adicionales de Prueba, solo puede acceder el Gerente.

exten => 1111,1,NoOp()
same => n,Progress()
same => n,MusicOnHold()

exten => 2222,1,NoOp()
same => n,Answer()
same => n,Set(CHANNEL(tonezone)=starwars)
same => n,Dial(Local/10,tT,30)

\end{lstlisting}

Aquí especificamos un contexto para el gerente, que incluye de forma jerarquíca todas las posibilidades que pudiesen tener los manager, y a su vez algunas extensiones de ejemplo, como la capacidad de escuchar la música en espera por defecto, o de llamarse a si mismo para comprobar que todo va bien.

Es importante considerar, que no hemos creado extensiones de marcación dentro del plan para cada una de los dispositivos SIP ya que como veremos adelante, este proceso se simplifica bastante utilizando Macros, como un sistema algo más avanzado del Dialplan de Asterisk.

Para recargar el Plan de Marcación y poder hacer uso de este desde la CLI de Asterisk:

\begin{lstlisting}[language=sh]
CLI> dialplan reload
\end{lstlisting}

 En este fichero vemos algunas cosas peculiares:

La aplicación NoOp sirve para hacer debug en consola. Si pusiéramos mensajes dentro del paréntesis, se verían en el log del CLI. Se pueden poner incluso variables del sistema, la cuestión es que esto viene más explicando en el apartado correspondiente de la documentación.

Lo verdaderamente significativo es la estructura. Como podemos ver, existen tres contextos (se encuentran dentro de corchetes). Como recordaremos, en el fichero sip.conf anterior, estos contextos hacen referencia a los contextos a donde las marcaciones que realizan los dispositivos irían. En este caso, podemos ver como el contexto [gerente] tiene un parámetro \textbf{include} que sirve para "incluir" todas las extensiones de los contextos a los que apunta el mismo. En este caso, incluiría por completo el contexto [manager], que este a su vez incluye el contexto [extensiones]. Esto significa que el gerente tiene "visibilidad" total sobre todo el plan de marcacion del sistema, en cambio un dispositivo de un empleado normal, solo podría acceder al plan de marcacion especifico dentro de su contexto, pero no del contexto del gerente.

Cuando se lo expliquemos al gerente, seguro que se le ocurren aplicaciones prácticas como el hecho de evitar que sus empleados llamen a números internacionales, o de tarificación especial.

\subsection{Comprobación: Configuración del Primer Plan de Marcación}

Esta parte es muy sencilla de comprobar su correcto funcionamiento.

Ahora utilizando nuestro Softphone X-Lite podríamos marcar la extensión 11 y el teléfono empezaría a sonar. Al descolgar se establecería la conexión.

Es posible que esta no se establezca. Para ello debemos utilizar el interfaz CLI para comprobar si el transcurso de la llamada esta produciendose de manera correcta. La salida que debemos ver en este flujo de información en detalle deber ser similar a la que mostramos en la siguiente Figura \ref{ejemplo_cli_sip}

\figura{ejemplo_cli_sip.png}{scale=1}{Ejemplo de una llamada en formato CLI}{ejemplo_cli_sip}{!ht}


\newpage

\subsection{Introduciendo al proveedor de Telefonía}

El último requisito que puso el gerente, es la posibilidad de tener 6 números de marcación directa, uno para cada departamento, y otro para él. Para poder conseguir esto, necesitamos contactar con un proveedor que nos ofrezca esto.

Como siempre, tenemos múltiples alternativas, pero si queremos evitar tener que comprar dispositivos, la más obvia es contactar con proveedores de Telefonía IP. Así que nos pusimos en contacto con una empresa muy conocida de la provincia, llamada \textbf{UCA Telecom}. Comparando los planes de precios que nos ofrecía '''FonicaTele''', el proveedor más conocido a nivel nacional, teníamos una oferta mucho más competitiva y encima, no teníamos la necesidad de andar contratando planes mensuales, ni comprando esos dispositivos que comentaba antes, para interconectar con la Red de Telefonía Conmutada clásica.

\begin{lstlisting}[language=bash,title={/etc/asterisk/sip.conf}]
[general]
register => ucaautos:12345678@home-asterisk.local:35060

[proveedorip]
defaultuser = ucaautos
fromuser = ucaautos
secret = 12345678
host = home-asterisk.local
port = 35060
type = friend
insecure = invite
qualify = yes
context = entrantes
\end{lstlisting}

Un dato muy relevante en la configuración de nuestro proveedor, es el hecho que el tipo es \textbf{friend} y no peer. Si nuestra intención fuera emitir llamadas a través de mismo, lo correcto sería esto. Pero como ademas, vamos a recibir llamadas entrantes dado que nos va a ofrecer un servicio de números Direct Dial-In (DDI), lo correcto es considerarlo, bidireccional con este tipo en concreto.

Pero de hecho, hay un parámetro que puede resultar aun más curioso de observar, que es el \textbf{insecure=invite}. Tenemos un problema si trabajamos con un dispositivo tipo friend: la autentificación bidireccional. Observamos que en la primera linea hay un mensaje register, para registrarnos en el proveedor ip. El tema es que el proveedor IP supuestamente, ha de autentificarse luego con nosotros cuando intenta enviarnos la llamada directa al DDI, pero nosotros al solo tener un dispositivo configurado, no podemos ofrecerle unas credenciales especificas (que probablemente sean las mismas que que las de llamadas salientes aunque no tiene porque ser así). Al habernos autentificado primero con el mensaje register, al poner el parámetro insecure de esta forma, como la conexión ya esta hecha (el primer mensaje REGISTER ya se envío), los sucesivos INVITE, no tendrían la necesidad de volver a autentificarse. Si quitamos el insecure, y activamos el debug de SIP, veremos como hay un error en la autentificación entrante, cuando tratamos de llamar a un número DDI.

Lo correcto en este caso, sería convertir este par friend, en un par tipo peer. Y por otro lado, crear un par tipo user, con el usuario de autentificación de entrada del proveedor, en caso que sea otro diferente (como es este caso). Con esto, ya no haría falta tener el parámetro insecure=invite activado, dado que todos los mensajes INVITE se autentificarían con el nuevo dispositivo tipo user, correctamente. Como este caso es relativamente raro (que los parámetros de autentificación de entrada y de salida sean diferentes para nuestro proveedor IP), entonces voy a dejarlo así, dando un voto de confianza a nuestro operador que a final de cuentas, es quien nos va a facturar a final de mes.

Introduciendo los siguientes cambios, obtenemos el resultado correcto de autentificación sin tener que exponer la máquina con el inseguridad de permitir mensajes INVITE sin autentificación:

\begin{lstlisting}[language=bash,title={/etc/asterisk/sip.conf}]
[proveedorip]
defaultuser = ucaautos
fromuser = ucaautos
secret = 12345678
host = home-asterisk.local
port = 35060
++type = peer
--type = friend
--insecure = invite
qualify = yes
--context = entrantes

++[ucatelecom]
++secret = 12345678
++host = home-asterisk.local
++port = 35060
++type = user
++context = entrantes
\end{lstlisting}

Considerar que \textbf{ucatelecom}, es el usuario al que conecta, la empresa UCAtelecom, y la contraseña casualmente es la misma que usamos para conectarnos a ellos casualmente, pero tampoco tendría porque ser así.

Por otro lado, hemos creado un contexto nuevo en el Dialplan, llamado [entrantes] haciendo referencia al lugar donde apunta nuestro nuevo proveedor SIP. Ahí vamos a definir todos los números DDI que nos ha solicitado el gerente, y reenviarlos con la función GoTo al contexto, extensión y prioridad correspondientes de cada departamento.

\begin{lstlisting}[language=bash,title={/etc/asterisk/extensions.conf}]
[entrantes]

exten => 956001101,1,Goto(extensiones,10,1)
exten => 956001111,1,Goto(extensiones,11,1)
exten => 956001121,1,Goto(extensiones,21,1)
exten => 956001131,1,Goto(extensiones,31,1)
exten => 956001141,1,Goto(extensiones,41,1)
exten => 956001151,1,Goto(extensiones,51,1)
\end{lstlisting}

Y con esto, ya estarían satisfechas las necesidades del gerente, por lo que concertamos una nueva cita con el Sr. Zutano para mostrarle que todo esta correctamente funcionando.

\subsection{Comprobación: Introduciendo al proveedor de Telefonía}

En este apartado tal y como estan definidos los ficheros de configuración todavía es imposible comprobar que todo funciona bien, dado que no tenemos definido en el plan de marcación un metodo para llamar a través del proveedor de telefonía IP. 

De momento, para probarlo a modo de ejemplo, podriamos establecer una ruta específica en la configuración específica del contexto del Gerente la siguiente linea:

\begin{lstlisting}[language=bash,title={/etc/asterisk/extensions.conf}]

[gerencia]

exten => 956001121,1,Dial(SIP/proveedorip/956001121}

\end{lstlisting}

Muy similar a la comprobación del apartado anterior, en este caso considerando por ejemplo, que tenemos configurada la extensión número 21, cuyo número Direct Dial-In (de marcación directa), es el 956001121, si llamamos a este número desde la extensión del gerente (10), debería de empezar a sonar en la extensión 21 el teléfono, dado que gracias a esta linea, estamos mandando llamada a traves del par que definimos en el fichero de configuración SIP llamado \textbf{proveedorip}, y luego el Proveedor, nos devuelve la llamada a traves del mismo par (recordemos que es un tipo \textbf{friend}, es decir "permite" tanto la entrada como la salida), y entraría según definido al contexto [entrantes]. En este momento, se redirigiría a la extensión 956001121, que esta a su vez, redirige a la extensión 21, dentro del contexto extensiones.